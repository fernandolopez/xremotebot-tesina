\chapter{Protocolo de capa de aplicación de XRemoteBot}\label{cha:protocolo}
Siendo una aplicación cliente-servidor XRemoteBot requiere algún método de
serialización para intercambiar datos entre los clientes y el servidor.
Considerando que el servidor de XRemoteBot es un servidor web que usa WebSockets
como protocolo de comunicación, el método de serialización propuesto es JSON
(Javascript Serialization Object Notation).
%En la sección~\ref{sec:alternativas}
%se describen algunas alternativas consideradas antes de tomar la decisión de
%evaluar JSON, BSON y CBOR en profundidad.

\section{Comparación entre JSON, BSON y CBOR}\label{sec:serializacion}

JSON cuenta con las siguientes características:
\begin{enumerate}
    \item Es un formato estandarizado por ECMA~\citep{ecma-404}
        y está especificado por la RFC-7159~\citep{rfc-7159}.
    \item Soporta los tipos de datos necesarios para intercambiar mensajes con
        los datos necesarios para controlar los robots usando un nivel de
        abstracción adecuado.
    \item Al ser un formato de texto es simple analizar el tráfico entre los
        clientes y el servidor para detectar posibles errores.
    \item Está soportado de forma nativa por los navegadores más
        utilizados.%\footnote{\url{http://caniuse.com/#search=json}}.
\end{enumerate}

Pero también existen otras alternativas cuyos objetivos son serialización
y de-serialización rápida de datos
como BSON (Binary JSON) y (CBOR Concise Binary Object Representation).

Algunas características de BSON y CBOR son:

\begin{enumerate}
    \item Codifican la información en formato binario.
    \item Son tan fáciles de usar como JSON.
    \item Ambos formatos pueden codificar y decodificar los mismos tipos de datos
        que JSON, además de soportar otros tipos de datos adicionales.
    \item CBOR está especificado por la RFC-7049~\citep{rfc-7049}.
    \item BSON tiene una especificación informal pero cuenta una
        implementación bien conocida siendo la representación de datos primaria en
        MongoDB\footnote{\url{http://bsonspec.org/}}.
    \item Por estar en formato binario decodificar una captura de tráfico
        para depurar un programa es más laborioso.
    \item En algunos casos estos formatos binarios generarán mensajes más
        chicos que JSON, pero no siempre.
    \item Requieren usar bibliotecas Javascript en los clientes web ya que los
        navegadores no lo implementan de forma nativa.
\end{enumerate}

Para tomar un decisión respecto al formato a utilizar se decidió generar
16 archivos \texttt{JSON} con datos aleatorios. Estos archivos se cargaron
con una cantidad de entradas múltiplo de 1024 desde 1024 hasta 16384, donde
cada una de estas entradas es un objeto \texttt{JSON} con 5 entradas de tipo
numérico, 5 strings, 5 objetos (cada uno con una entrada numérica) y 5
arrays (cada uno con 2 entradas de tipo string).

Estos archivos se cargaron desde scripts similares en Python y Javascript que
de-serializaron y serializaron estos datos repetidas veces calculando el tiempo
promedio que llevó hacer cada una de estas acciones para cada formato.

Se eligió hacer las pruebas con Python y Javascript porque el primero es el
lenguaje de implementación del servidor y del cliente que probablemente tenga
más uso en un futuro, mientras que Javascript es el lenguaje de implementación
del cliente que se ejecutará en los navegadores web y que puede servir como
base para implementaciones, por ejemplo, con
Brython\footnote{\url{http://www.brython.info/}},
Skulp\footnote{\url{http://www.skulpt.org/}} u
Opal\footnote{\url{http://opalrb.org/}}.

Para hacer el experimento repetible, la implementación en
Javascript se ejecutó con el intérprete \textbf{nodejs} basado en
el motor \textbf{v8} usado por \textbf{Chrome} y además se crearon
scripts de apoyo para generar los datos de prueba, ejecutar los scripts
con los distintos formatos de forma automatizada y formatear los datos
resultantes en archivos CSV.

Todas las pruebas se realizaron sobre Lihuen 6 beta (basado en Debian Jessie)
en una notebook con procesador ``Intel(R) Core(TM) i3 CPU M 370 @ 2.40GHz''
y 4GB de RAM % NOTA: son 4GB no 4GiB
usando Python 3.4.2 y NodeJS 0.10.35\footnote{Se usó NodeJS 0.10.35 ya que
la versión 0.10.29 distribuida al momento con Debian Jessie tenía un bug que
generaba un error de memoria al
deserializar strings largos con \texttt{JSON.load} (CVE-2014-5256)}

En estas pruebas se tomaron mediciones de los tiempos de serialización y deserialización
en Python para las 3 alternativas en relación a la cantidad de entradas
(figura~\ref{fig:ser-time-py}), los tiempos de serialización y deserialización en
Javascript para las 3 alternativas también en relación a la cantidad de entradas
(figura~\ref{fig:ser-time-js}) y finalmente la relación entre la cantidad de
entradas y el tamaño de los strings generados al serializar los mismos datos en los 3 formatos
evaluados
(figura~\ref{fig:ser-size}).

\begin{figure}
    \centering
    \begin{framed}
        \begin{tikzpicture}[trim axis left, trim axis right, scale=0.7]
            \begin{axis}[
                    height=10cm,
                    width=10cm,
                    scaled x ticks = false, % No usar notación exponencial
                    xlabel=Cantidad de entradas,
                    ylabel=Segundos,
                    x tick label style={/pgf/number format/fixed, rotate=45, anchor=north east}, % Las etiquetas son números y rotar 45°
                    xtick=data, % Generar una etiqueta por punto en los datos
                    ylabel near ticks, % Poner las etiquetas más o menos cerca de algunos datos
                    ymin=0, % No hay tiempos negativos
                    xmin=1024, % La escala empieza en 1024
                    legend style={legend pos=north west}, % Cuadro de leyendas a la derecha arriba
                    every axis x label/.style={
                        at={(axis description cs:0.5,-0.15)}, % Desplazar la descripción del eje x
                        anchor=north, % El desplazamiento relativo a la parte superior del texto
                    },
                    legend cell align=left, % Alineación del texto en la caja de leyendas
                ]
                \foreach \s in {json, bson, cbor} {
                    \addlegendentryexpanded{\s\ dump}
                    \addplot table [x=entries, y=dump_time, col sep=comma] {data/py-\s.csv};

                    \addlegendentryexpanded{\s\ load}
                    \addplot table [x=entries, y=load_time, col sep=comma] {data/py-\s.csv};
                }
            \end{axis}
        \end{tikzpicture}
    \end{framed}

    \caption{Cantidad de entradas versus tiempo de serialización y deserialización en Python}
    \label{fig:ser-time-py}
\end{figure}

\begin{figure}
    \centering
    \begin{framed}
        \begin{tikzpicture}[trim axis left, trim axis right, scale=0.7]
            \begin{axis}[
                    height=10cm,
                    width=10cm,
                    scaled x ticks = false, % No usar notación exponencial
                    xlabel=Cantidad de entradas,
                    ylabel=Segundos,
                    x tick label style={/pgf/number format/fixed, rotate=45, anchor=north east}, % Las etiquetas son números y rotar 45°
                    xtick=data, % Generar una etiqueta por punto en los datos
                    ylabel near ticks, % Poner las etiquetas más o menos cerca de algunos datos
                    ymin=0, % No hay tiempos negativos
                    xmin=1024, % La escala empieza en 1024
                    legend style={legend pos=north west}, % Cuadro de leyendas a la derecha arriba
                    every axis x label/.style={
                        at={(axis description cs:0.5,-0.15)}, % Desplazar la descripción del eje x
                        anchor=north, % El desplazamiento relativo a la parte superior del texto
                    },
                    legend cell align=left, % Alineación del texto en la caja de leyendas
                ]
                \foreach \s in {json, bson, cbor} {
                    \addlegendentryexpanded{\s\ dump}
                    \addplot table [x=entries, y=dump_time, col sep=comma] {data/js-\s.csv};

                    \addlegendentryexpanded{\s\ load}
                    \addplot table [x=entries, y=load_time, col sep=comma] {data/js-\s.csv};
                }
            \end{axis}
        \end{tikzpicture}
    \end{framed}
    \caption{Cantidad de entradas versus tiempo de serialización y deserialización en Javascript (nodejs)}
    \label{fig:ser-time-js}
\end{figure}


\begin{figure}
    \centering
    \begin{framed}
        \begin{tikzpicture}[trim axis left, trim axis right, scale=0.7]
            \begin{axis}[
                    height=10cm,
                    width=10cm,
                    scaled x ticks = false,
                    xlabel=Cantidad de entradas,
                    ylabel=Tamaño en MiB,
                    x tick label style={/pgf/number format/fixed, rotate=45, anchor=north east},
                    xtick=data,
                    ylabel near ticks,
                    ymin=0,
                    xmin=1024,
                    legend style={legend pos=north west},
                    every axis x label/.style={
                        at={(axis description cs:0.5,-0.15)},
                        anchor=north,
                    },
                    legend cell align=left
                ]
                \foreach \s in {json, bson, cbor} {
                    \addlegendentryexpanded{\s}
                    \addplot table [x=entries, y=serialized_size, col sep=comma] {data/js-\s.csv};

                }
            \end{axis}
        \end{tikzpicture}
    \end{framed}
    \caption{Cantidad de entradas versus tamaño del archivo serializado}
    \label{fig:ser-size}
\end{figure}

De la figura~\ref{fig:ser-size} se desprende que los distintos métodos de
serialización no ofrecen diferencias significativas en el tamaño de los
strings generados para los volúmenes de datos probados. Por otro lado,
se puede observar en la figura~\ref{fig:ser-time-py} que, al menos en
Python y con la biblioteca usada, CBOR tiene los mejores tiempos. Sin
embargo en las pruebas con Javascript en la figura~\ref{fig:ser-time-js}
se puede observar que los mejores tiempos son para el formato JSON.
Esto es entendible ya que este es el único formato (entre los evaluados) procesado de forma
nativa en este lenguaje\footnote{Si bien es posible encontrar parsers nativos para los
otros formatos y usarlos desde NodeJS no se utilizó esta modalidad porque
no sería una prueba realista dado que los motores Javascript de los
navegadores no permiten esto.}.

Teniendo en cuenta que las diferencias de tiempo al procesar los datos
en Python son, en comparación con la versión en Javascript, marginales
(apenas alrededor de $0.25$ segundos para $2^{14}$
entradas entre el mejor y peor tiempo de deserialización) y las diferencias
en cuanto a tamaño del string generado al
serializar también son pequeñas (aproximadamente 2MiB de diferencia para
$2^{14}$ entradas entre el mejor y el peor método). Se eligió implementar
el protocolo de XRemoteBot con JSON, que en estas pruebas cuenta con los
mejores tiempos de serialización y deserialización en Javascript, además
de estar soportado en los 3 lenguajes sin necesidad de utilizar bibliotecas
externas.


\section{Protocolo diseñado para XRemoteBot}

Como se describió anteriormente, XRemoteBot utiliza el protocolo WebSocket como
protocolo de la capa de transporte y mensajes codificados con JSON como
protocolo de la capa de aplicación. El propósito de este capítulo es describir
este protocolo de capa de aplicación y detallar los motivos por los cuales
se modeló el protocolo de esta manera.

% FIXME agregue
\subsection{Consideraciones generales}


El protocolo fue diseñado pensando que el mismo debería representar los
distintos comandos enviados a los robots como si fueran mensajes a objetos
del paradigma de programación orientado a objetos. Este diseño por un lado
mapea de forma muy directa los mensajes de la API
con los métodos que un programador utilizaría para controlar a los robots
usando la biblioteca DuinoBot,  y por otro lado, provee un alto grado de
flexibilidad a la hora de agregar nuevos tipos de mensajes a la API sin
necesidad de reescribir grandes porciones de código.

En algunos de los lenguajes orientados a objetos más populares como Python
o Ruby, al ejecutar un método pueden ocurrir básicamente dos cosas:
se puede obtener un valor de resultado o bien un error
en tiempo de ejecución en la forma de una excepción. Para modelar este
comportamiento el servidor XRemoteBot cuenta con dos tipos de mensajes
de respuesta:

\begin{itemize}
    \item El mensaje tipo ``value'' que retorna el valor resultante de
        ejecutar el método invocado.
    \item El mensaje tipo ``error'' que representa el error en tiempo
        de ejecución que típicamente se representa con excepciones.
\end{itemize}

Los mensajes de error de XRemoteBot tienen como contenido
un mensaje de texto que describe el error producido. Los clientes
pueden tomar este mensaje
de error y mostrarlos en pantalla o usarlos como descripción de una
excepción como se hace en el cliente Python por ejemplo.

\subsection{Alternativas analizadas}\label{sec:alternativas}
En definitiva cualquier forma de serialización de datos hubiera
sido suficiente para este protocolo. Se eligió evaluar protocolos
similares a JSON dada su facilidad de uso, facilidad de lectura
y la disponibilidad
de funciones para manipular JSON en las bibliotecas estándar
de los tres lenguajes elegidos. Esto es importante porque es
la intención del autor que cualquiera con conocimientos básicos
de programación pueda desarrollar clientes en otros lenguajes.

Estas características no se dan en formatos como YAML que
son más complejos y solamente no está soportado por la biblioteca
estándar de Python ni por Javascript. Tampoco cumple con las características
requeridas XML que si bien es posible
procesarlo en cualquiera de estos lenguajes sin agregar bibliotecas
externas, un posible desarrollador de un cliente tendría más
problemas para implementar el cliente ya que las API para
manipular XML suelen
ser más complejas dada la estructura de estos archivos y además
la información codificada en XML es más difícil de leer por
ejemplo en una captura de red.

También se descartó crear un mecanismo de serialización específico
para este trabajo, ya que de hacerlo debería haber implementado
todo el protocolo en tres lenguajes distintos y además cualquiera
que quisiera crear un cliente para XRemoteBot debería también
implementar el protocolo en el lenguaje deseado, con toda
la complejidad que tiene validar y extraer información con
distintos tipos de datos de forma correcta y segura.

%FIX ME agregue
\subsection{El protocolo basado en JSON}

Como se mencionó anteriormente, tanto para los parámetros, como para los valores de respuesta, los tipos
de datos representables de forma nativa con JSON resultan suficientes para
implementar una API similar a la de DuinoBot.

Internamente XRemoteBot
convierte los valores, objetos y arrays de JSON a distintos tipos de
datos de Python de la forma descripta en la
figura~\ref{tbl:rel_json_python}, siguiendo, en general las transformaciones
realizadas por defecto por las funciones de procesamiento de el módulo
JSON de la biblioteca estándar de
Python\footnote{\url{https://docs.python.org/3/library/json.html}}.

\begin{figure}
    \centering
    \begin{tabu}{l|l}
        JSON & Python \\
        \hline
        \texttt{string} & \texttt{str} \\
        \texttt{number} & \texttt{int} o \texttt{float}\footnote{Dependiendo
        del mensaje invocado
        puede ser interpretado como \texttt{int} o \texttt{float}.}\\
        \texttt{object} & \texttt{dict} \\
        \texttt{array}  & \texttt{list} \\
        \texttt{true}/\texttt{false} & \texttt{True}/\texttt{False} (bool) \\
        \texttt{null} & \texttt{None} \\
    \end{tabu}
    \caption{Relación entre los tipos y valores de JSON y los usados en
    Python}
    \label{tbl:rel_json_python}
\end{figure}

\subsection{Mensajes del cliente al servidor}

Los mensajes al servidor son objetos JSON con dos campos obligatorios y dos
opcionales. Los campos que componen a estos objetos  son:

\begin{description}
    \item[\texttt{entity}:] cuyo valor es un string que describe conceptualmente
        el receptor del mensaje (obligatorio).
    \item[\texttt{method}:] cuyo valor es un string con el nombre del método a
        invocar (obligatorio).
    \item[\texttt{args}:] cuyo valor es un array posiblemente vacío con los
        argumentos que deben pasarse al método invocado (opcional).
    \item[\texttt{msg\_id}:] si está presente el servidor incorpora este
        campo con su valor original en la respuesta. Sirve para identificar a
        qué petición
        corresponde cada respuesta en implementaciones asincrónicas
        como la de Javascript.
\end{description}

En la actual implementación los valores posibles del campo \texttt{entity} son
\texttt{``global''} y \texttt{``robot''}. El primero
representa acciones globales como el login y el último representa una
abstracción de los robots soportados por XRemoteBot.

La entidad ``global'' soporta los métodos:
\begin{description}
    \item[\texttt{authentication\_required}:] No recibe argumentos. Retorna el valor 
        \texttt{true} si el servidor está configurado para requerir
        autenticación y \texttt{false} en caso contrario.
    \item[\texttt{authenticate}:] Recibe un string como argumento. Si el string se corresponde
        con la \texttt{api\_key} (una clave generada de forma aleatoria que identifica
        a un usuario específico) de algún usuario
        y esa \texttt{api\_key}  no
        ha expirado retorna el valor \texttt{true} y \texttt{false} en caso contrario.
\end{description}

Entre otros la entidad ``robot'' soporta los métodos:
\begin{description}
    \item[\texttt{backward}:] Recibe un objeto que identifica a un robot
        específico, una velocidad y un tiempo en segundos. Mueve el robot hacia
        atrás el tiempo indicado y luego contesta al cliente un valor \texttt{null}.
        Si no se envía el parámetro de tiempo el robot se mueve
        de forma indefinida.
    \item[\texttt{forward}:] Recibe un objeto que identifica a un robot
        específico, una velocidad y un tiempo en segundos. Mueve el robot hacia
        adelante el tiempo indicado y luego contesta al cliente un valor \texttt{null}.
        Si no se envía el parámetro de tiempo el robot se mueve
        de forma indefinida.
    \item[\texttt{turnLeft}:] Recibe un objeto que identifica a un robot
        específico, una velocidad y un tiempo en segundos. Gira hacia
        la izquierda el tiempo indicado y luego contesta al cliente un valor \texttt{null}.
        Si no se envía el parámetro de tiempo el robot se mueve
        de forma indefinida.
    \item[\texttt{turnRight}:] Recibe un objeto que identifica a un robot
        específico, una velocidad y un tiempo en segundos. Gira hacia
        la derecha el tiempo indicado y luego contesta al cliente un valor \texttt{null}.
        Si no se envía el parámetro de tiempo el robot se mueve
        de forma indefinida.
    \item[\texttt{getObstacle}:] Recibe un objeto que identifica a un robot y,
        opcionalmente una distancia. Retorna el valor 
        \texttt{true} si hay un obstáculo a una distancia menor o igual a la
        indicada. Si no se envía la distancia se asume un valor por defecto,
        este valor por defecto depende del modelo de robot usado ya que los
        sensores tienen rangos diferentes.
    \item[\texttt{getLine}:] Recibe un objeto que identifica a un robot y
        retorna un \texttt{array}
        con los valores de los sensores de línea del robot.
\end{description}


Opcionalmente un modelo dado de robot puede soportar otros métodos o retornar
error en alguno de los métodos anteriores. El protocolo es fácilmente
extensible ya que simplemente modela un pasaje de mensajes con argumentos,
para agregar nuevos mensajes basta con enviar distintos strings en el
campo ``method'' y para soportar nuevos tipos de entidades simplemente
se puede enviar un string diferente en ``entity''.

Por ejemplo, si se requiriese implementar
un sistema que controle la iluminación de una casa,
se podrían definir los métodos: ``main\_bedroom'', ``children\_bedroom'',
``kitchen'', ``living'' y ``garage''. Si cada uno de éstos
controla la luz de la habitación descripta por su nombre, podrían recibir
como
argumento la intensidad de luz deseada como un número de 0 a 255. Finalmente
podemos agrupar estos métodos en una ``entity'' con el nombre``houselight''.

Podríamos controlar las luces con una secuencia de mensajes como
los mostrados en el código~\ref{lst:houselight}. Por supuesto, luego
habría que agregar soporte para estas funcionalidades en el servidor
y los clientes pero esto tampoco es
un problema significativo en XRemoteBot.

\begin{lstlisting}[language=python,
caption={Ejemplo de una posible extensión al protocolo para controlar
las luces de una casa},
label=lst:houselight]
{
    "entity": "houselight",
    "method": "main_bedroom",
    "args": [60],
}
{
    "entity": "houselight",
    "method": "garage",
    "args": [0],
}
{
    "entity": "houselight",
    "method": "living",
    "args": [255],
}
\end{lstlisting}


Los métodos bloqueantes como \texttt{backward} (que con DuinoBot demoran
la ejecución del proceso hasta que transcurra el tiempo indicado como
parámetro) son modelados desde el servidor demorando la respuesta al método
gracias a que el framework web Tornado (usado para dar soporte al servidor
para los protocolos HTTP y WebSockets, entre otras utilidades)
funciona con un ``bucle principal''
(``main loop'') donde se pueden encolar trabajos para que se ejecuten
en el futuro.
Esta funcionalidad no es necesaria para la mayoría de los clientes que
pueden implementar las demoras de forma local usando la función
\texttt{sleep()}\footnote{Tanto en Ruby como en Python la función \texttt{sleep()} detiene
el hilo actual la cantidad de segundos que indique su único parámetro.},
pero es necesario para la implementación en Javascript donde no es posible
demorar la ejecución del script sin impactar de forma negativa el rendimiento
del navegador (al menos en la pestaña actual).



\section{Mensajes del servidor a los clientes}

El servidor responde a los clientes con mensajes que contienen una única
entrada \texttt{response}. Esta entrada identifica el tipo de respuesta
que puede ser \texttt{value} o \texttt{error}. Si el cliente envía un
campo \texttt{msg\_id} en la petición, el servidor incorporará además
un campo \texttt{msg\_id} idéntico en la respuesta correspondiente
a esa petición (como se mencionó anteriormente para soportar
clientes asincrónicos de forma correcta).

\subsection{Mensajes tipo value}

Los mensajes \texttt{value} representan los valores de respuesta de
los métodos invocados.

Los campos de los mensajes tipo \texttt{value} son:

\begin{description}
    \item[\texttt{response}:] identifica el tipo de mensaje. En este caso
        es un string con el valor \texttt{``value''}.
    \item[\texttt{value}:] el valor retornado por el método (puede ser
        cualquier valor soportado por JSON).
    \item[\texttt{msg\_id}:] si la petición contiene el campo \texttt{msg\_id} se
        copia el mismo campo, sino este campo se omite.
\end{description}

Para los mensajes que requieren
un tiempo de demora el servidor contestará con el mensaje \texttt{value}
una vez transcurrido ese tiempo. Para no demorar la atención de peticiones
de otros clientes el servidor utiliza las funcionalidades provistas por
Tornado para atender peticiones de forma asíncrona~\citep{dory_2012}.

\subsection{Mensajes tipo error}

Los mensajes \texttt{error} representan eventos anómalos que
generalmente en un sistema no distribuido se modelarían utilizando
excepciones, por ejemplo errores de codificación, peticiones a recursos
ocupados, etc...

Campos de los mensajes \texttt{error}:
\begin{description}
    \item[\texttt{response}:] identifica el tipo de mensaje, en este caso
        es un string con el valor \texttt{``error''}.
    \item[\texttt{message}:] un mensaje de error descriptivo.
    \item[\texttt{msg\_id}:] si la petición tenía \texttt{msg\_id} se
        copia el mismo, sino este campo se omite.
\end{description}


\section{Ejemplos de interacción entre los clientes y el servidor}

Al comenzar la conexión entre el cliente y el servidor, se intercambian
los primeros mensajes donde el cliente pregunta al servidor si el
mismo requiere autenticación y si es así envía la \textit{API key}
correspondiente. En la figura~\ref{fig:ej_autenticacion} se detalla
el intercambio de mensajes entre un cliente y el servidor en un
caso de autenticación exitoso con un cliente que envía un \texttt{msg\_id}.
Como se puede ver el \texttt{msg\_id}
no tiene por qué ser consecutivo con el del mensaje anterior, incluso
puede llegar a ser un string, las únicas restricciones sobre el
\texttt{msg\_id} son que tiene que ser copiado por el servidor
en las respuestas y que el valor debe poder ser utilizado como
identificador de un atributo (usando notación con corchetes) en un
objeto Javascript. Es decir puede ser un número o un string
arbitrario\footnote{\url{https://developer.mozilla.org/en-US/docs/Web/JavaScript/Reference/Operators/Property\_Accessors}}.

\begin{figure}
    \centering
    \includegraphics[scale=0.3]{figures/mensajes}
    \caption{Ejemplo de secuencia de autenticación}
    \label{fig:ej_autenticacion}
\end{figure}

Otra secuencia de operaciones típica para XRemoteBot es mover un robot
y luego pedir los valores de alguno de sus sensores, en la
figura~\ref{fig:ej_movimiento_y_sensor} se puede ver un caso donde se
mueve el robot hacia adelante durante 1 segundo y luego consulta si
existe algún obstáculo a 10 centímetros o menos del robot, en el primer
argumento de los mensajes
enviados a la entidad \textit{robot} se puede ver el objeto que identifica
al robot específico a utilizar.

\begin{figure}
    \centering
    \includegraphics[scale=0.3]{figures/mensajes_robot}
    \caption{Ejemplo de movimiento y acceso a sensores de un robot}
    \label{fig:ej_movimiento_y_sensor}
\end{figure}

Por último, en la figura~\ref{fig:ej_error} se puede ver un mensaje de error
típico cuando el cliente envía una petición mal formada donde no existe
el campo \textit{method} obligatorio, notar que este mensaje a diferencia
de los anteriores no tiene un campo \textit{msg\_id} pero esto no causaría
ningún error ya que este campo no es obligatorio, el único error en la
petición es la ausencia de \textit{method}.

\begin{figure}
    \centering
    \includegraphics[scale=0.3]{figures/mensajes_error}
    \caption{Ejemplo de un mensaje de error ante una petición mal formada}
    \label{fig:ej_error}
\end{figure}

\section{Modalidades del servidor}

A fin de hacer que el servidor pueda exponerse al público se cuenta con
un sistema con autenticación por \textit{API key}, pero estas claves
son largas resultando difícil recordarlas y escribirlas sin errores. En
ámbitos locales, como puede ser una red wifi en un aula, este mecanismo
de autenticación puede resultar molesto e innecesario, por lo que el
servidor es configurable de forma tal que se puede deshabilitar este
sistema de autenticación. Para que el servidor opere sin requerir
una \textit{API key} a los clientes basta con configurar la opción
\texttt{public\_server} en \texttt{False} como se indica en la
sección~\ref{sec:configuraciones}. Esto también deshabilita
el sistema de reservas, por lo que cualquier usuario puede usar
cualquier robot en este modo.

Dado que existe esta modalidad el protocolo cuenta con un mensaje
para preguntar si es necesario proseguir con la autenticación
(\texttt{authentication\_required}), con lo cuál se pueden crear
scripts que se adapten a ambos escenarios.

%Por otro lado, si los robots se encuentran en el mismo lugar
%que los usuarios, puede deshabilitarse el streaming de video
%con la opción \texttt{disable\_streaming}.
%
%De esta manera, si el servidor se encuentra disponible al público
%a través de Internet, es recomendable usar la configuración mostrada
%en el código~\ref{lst:conf-internet}. En cambio si el servidor es
%utilizado en una red local, es recomendable usar la configuración
%mostrada en el código~\ref{lst:conf-lan}.
%
%\begin{lstlisting}[language=Python,
%caption={Configuración recomendada para servidores públicos},
%label=lst:conf-internet]
%authentication_required = True
%disable_streaming = False
%\end{lstlisting}
%\begin{lstlisting}[language=Python,
%caption={Configuración recomendada para servidores locales},
%label=lst:conf-lan]
%authentication_required = False
%disable_streaming = True
%\end{lstlisting}
