% $Log: abstract.tex,v $
% Revision 1.1  93/05/14  14:56:25  starflt
% Initial revision
% 
% Revision 1.1  90/05/04  10:41:01  lwvanels
% Initial revision
% 
%
%% The text of your abstract and nothing else (other than comments) goes here.
%% It will be single-spaced and the rest of the text that is supposed to go on
%% the abstract page will be generated by the abstractpage environment.  This
%% file should be \input (not \include 'd) from cover.tex.
En esta tesina diseñé e implementé un servidor y varios clientes que permiten
programar robots didácticos de forma remota compartiendo un único enlace
físico entre el servidor y los robots y visualizar los movimientos de los
mismos también de forma remota. El servidor implementa un protocolo de capa
de aplicación en base a objetos JSON a través de una conexión WebSocket que
permite a los clientes interactuar con distintos modelos de robots con gran
flexibilidad, moverlos y acceder a los valores de sus sensores, además de
facilitar la visualización de los mismos a través de streaming de video.
Implementé clientes en Javascript (a ser usado desde el navegador),
Python y Ruby que permiten que este proyecto sea utilizado para enseñar
cualquiera de estos lenguajes de programación, además de diseñar y documentar
el protocolo de capa de aplicación para que sea relativamente sencillo
implementar otros clientes.

