\chapter*{Prólogo}\label{cha:prologo}


Esta tesina se enmarca en el proyecto \proyecto{} del
LINTI\footnote{\url{http://linti.unlp.edu.ar}}
y el desarrollo realizado
es una contribución a dicho proyecto.
Se implementó un servidor denominado XRemoteBot y tres
clientes compatibles
con el mismo escritos en tres lenguajes distintos. El motivo y las elecciones hechas
en el desarrollo del servidor y los clientes serán relatados a lo largo de este
trabajo.

A continuación se describe la forma en que se ha organizado este informe y en el CD
adjunto se entrega tanto una copia del mismo como los fuentes desarrollados.
Aunque
cabe aclarar que todo el trabajo implementado ha sido desarrollado con software libre,
será liberado bajo licencia
MIT\footnote{\url{http://opensource.org/licenses/MIT}} y se puede descargar
desde GitHub\footnote{%
\url{https://github.com/fernandolopez/xremotebot} y
\url{https://github.com/fernandolopez/xremotebot-clients}}.

El capítulo~\ref{cha:motivacion} relata la motivación para el desarrollo de
XRemoteBot.

El capítulo~\ref{cha:arte} presenta un relevamiento de desarrollos y
proyectos similares.

El capítulo~\ref{cha:myro_y_duinobot} describe el modo funcionamiento de las
bibliotecas usadas hasta el momento en el proyecto \proyecto{} para
interactuar con los robots, estas
bibliotecas son la base de XRemoteBot.

El capítulo~\ref{cha:xremotebot} describe el modo de funcionamiento y
tecnologías utilizadas en el servidor
XRemoteBot desarrollado para esta tesina y de su antecesor RemoteBot.

El capítulo~\ref{cha:clientes} describe el modo de uso, modo de
funcionamiento y
tecnologías utilizadas en los tres clientes desarrollados para
XRemoteBot, especialmente el cliente Javascript que es el más complejo
en su implementación.

El capítulo~\ref{cha:protocolo} describe el protocolo de capa de aplicación diseñado para
el servidor desarrollado y las modalidades de operación del mismo.

El capítulo~\ref{cha:pruebas} describe el entorno en el que fue probado
XRemoteBot, detallando algunas de pruebas realizadas y algunas
modificaciones
realizadas al proyecto en base a los resultados de estas pruebas.

El capítulo~\ref{cha:conclusiones} contiene las conclusiones del trabajo
y el trabajo a futuro.
