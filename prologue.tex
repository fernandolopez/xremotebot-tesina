\chapter*{Prólogo}\label{prologo}

XRemoteBot es un servidor web que provee una API JSON para interactuar
con robots didácticos de forma remota y compartiendo un único enlace
físico con los robots. Está implementado en Python usando principalmente el
framework web Tornado, SQLAlchemy para acceder a la base de datos
%, Celery
%para manejar las tareas que pueden generar una demora innecesaria en el
%procesamiento de requerimientos HTTP
y los módulos DuinoBot y Myro para
la comunicación con los robots. El cliente Javascript se encuentra
integrado con el servidor y la intensión es que sea utilizado desde una
vista web provista por el mismo, mientras que los clientes Python y Ruby
son independientes y la intensión es que sean utilizados en programas
tradicionales que se ejecuten desde fuera del navegador para enseñar
a programar en estos lenguajes usando un entorno habitual. XRemoteBot
es un rediseño y una reescritura completa de RemoteBot, un servidor
simple realizado como aplicación auxiliar de una aplicación Android
desarrollada
como trabajo práctico para la materia Laboratorio de Software.

El capítulo 1 relata la motivación para el desarrollo de XRemoteBot
y alternativas similares en existencia en la actualidad.

El capítulo 2 presenta el estado del arte.

El capítulo 3 describe la arquitectura de DuinoBot, Myro, RemoteBot y
XRemoteBot detallando las mejoras provistas por XRemoteBot.

El capítulo 4 describe el protocolo de capa de aplicación diseñado para
este servidor y las modalidades de operación del servidor.

El capítulo 5 describe la implementación de los clientes y el modo de uso
de cada uno destacando algunas diferencias y decisiones de diseño en la
implementación en Javascript.

El anexo A muestra las pruebas realizadas para seleccionar un método
de serialización para el protocolo de capa de aplicación desarrollado.

El anexo B detalla las características por las cuales se eligió utilizar
WebSockets y Promises, y las consecuencias de estas elecciones.

