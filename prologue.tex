\chapter*{Prólogo}\label{cha:prologo}

%fixme ACA EMPEZARIA CON ALGO TIPO: Esta tesina se enmarca en el proyecto xxxx y el desarrollo realizado es una contribución a dicho proyecto.
% A continuación se describe la forma en que se ha organizado este informe y en el CD/DVD adjunto se entrega tanto copia del mismo como de los fuentes desarrollados, aunque cabe aclarar que todo el trabajo implemetado ha sido desaroollado con SL y liberado bajo gpl??? jajaj y se lo puede descargar de.xxxxx

Esta tesina se enmarca en el proyecto \proyecto{} y el desarrollo realizado
es una contribución a dicho proyecto.
El desarrollo realizado para este trabajo consta del servidor XRemoteBot y tres
clientes compatibles
con el mismo escritos en tres lenguajes distintos, el motivo y las elecciones hechas
en el desarrollo de el servidor y los clientes serán relatados a lo largo de este
trabajo.

A continuación se describe la forma en que se ha organizado este informe y en el CD
adjunto se entrega tanto una copia del mismo como los fuentes desarrollados, aunque
cabe aclarar todo el trabajo implementado ha sido desarrollado con software libre,
será liberado bajo licencia
MIT\footnote{\url{http://opensource.org/licenses/MIT}} y se podrá descargar desde
\url{https://github.com/fernandolopez/xremotebot} y
\url{https://github.com/fernandolopez/xremotebot-clients}.

El capítulo~\ref{cha:motivacion} relata la motivación para el desarrollo de XRemoteBot
y las  alternativas similares en existencia en la actualidad.

El capítulo~\ref{cha:arte} presenta un relevamiento de desarrollos y proyectos similares.

El capítulo~\ref{cha:arquitectura} describe las arquitecturas planteadas por las distintas APIs con las que se trabaja: DuinoBot, Myro, RemoteBot y XRemoteBot,  detallando las mejoras propuestas por XRemoteBot.

El capítulo~\ref{cha:protocolo} describe el protocolo de capa de aplicación diseñado para
el servidor desarrollado y las modalidades de operación del mismo.

El capítulo~\ref{cha:clientes} describe la implementación de los distintos clientes desarrollado y el modo de uso
de cada uno, destacando algunas diferencias y decisiones de diseño en la
implementación en Javascript.

%FIXME ALGUNAS PRUEBAS Y CONCLUSOINES

El anexo~\ref{cha:serializacion} muestra las pruebas realizadas para seleccionar un método
de serialización para el protocolo de capa de aplicación desarrollado.

%FIXME NO SE SI ESTO NO DEBERIA ESTAR DENTRO DEL CAPITULO DONDE SE DESCRIBE LA IMPLEMENTACION.. EN UNA SECCIPON TIPO ANALISIS DE ALTERNATIVAS.. O ALGO ASI
El anexo~\ref{cha:websockets_y_promises} detalla las características por las cuales se eligió utilizar
WebSockets y Promises, y las consecuencias de estas elecciones.

