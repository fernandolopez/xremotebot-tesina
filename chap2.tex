\chapter{Controlando dispositivos de forma remota}\label{cha:arte}

Existen diferentes alternativas para controlar robots o microcontroladores
de forma remota. Muchas de estas se agrupan bajo la denominación
``Internet of Things''\footnote{\url{https://www.iotwf.com/resources/1}}.
En este capítulo se describirán algunas de las
más destacadas entre las más similares a los requerimientos planteados en
esta propuesta.


\section{Educabot}
El proyecto Educabot\footnote{\url{http://www.educabot.org/}} tiene por
objetivo enseñar tecnología a niños y adultos a través
del uso, programación y construcción de robots. En el sitio del proyecto
se ofertan cursos orientados a los distintos niveles y 
en la última conferencia de Python de Argentina (Pyconar 2014) uno de
los cofundadores del proyecto mencionó que se llevan adelante actividades
con los robots
en distintas escuelas de la Ciudad Autónoma de Buenos
Aires\footnote{\url{https://youtu.be/1oCOAtX9OS4}}.

% FIXME: ES ORIGINARIO DE ..... EN ESCUELAS? EN ONGS? DONDE SE LOS USA?
% Son cursos privados según dijo el tipo, pero no tengo de donde citar
% eso no dice nada el sitio.
% FIXME: En el sitio no mencionan nada de los cursos en escuelas lo único
% que encontré es lo de youtube que es una charla en la Pycon del fundador

En la parte de construcción, este proyecto plantea un modelo de robot
denominado
``Rolo'' para los jóvenes de más de 10 años, mientras que para los más chicos
se plantean actividades con el robot ``elBrian'' que incluyen
controlarlo a través de una interfaz web que muestra las imágenes emitidas
por la cámara incorporada en este robot y además permite controlarlo con
botones en pantalla que determinan en qué dirección debe moverse el robot
(figura~\ref{fig:elbrian}).

\begin{figure}
    \centering
    \includegraphics[width=0.5\textwidth]{figures/elbrian-1}
    \caption{Pantalla principal de elBrian (en el recuadro rojo normalmente
        se muestra la cámara de el robot)}
    \label{fig:elbrian}
\end{figure}

Las tecnologías utilizadas en la interfaz web de ``elBrian'' coinciden en gran
parte con las utilizadas en el desarrollo de XRemoteBot, pero el objetivo del
servidor de ``elBrian'' es controlar un único robot en un ambiente local.
Además este servidor
se instala en el robot utilizado,  cuestión que sería imposible en los
robots basados en
microcontroladores AVR y PIC (``Basic Stamp 2`` de Parallax)
a los que XRemoteBot se encuentra dirigido.

%FIXME: fijate de poner aunque sea notas al pie con las referencias de los paquetes o protocolos que nmbras... el que no sabe.. se muere...

El servidor web de ``elBrian'' está implementado usando el framework
Tornado\footnote{\url{http://www.tornadoweb.org/}},
WebSockets\footnote{\url{http://www.websocket.org/}},
mjpg-streamer\footnote{\url{https://github.com/jacksonliam/mjpg-streamer}},
opencv\footnote{\url{http://opencv.org/}}
y JSON\footnote{\url{http://json.org/}}.
El mismo está diseñado para ejecutarse
en el robot ya que el mismo está basado en una placa RaspberryPi, la cuál
cuenta con un procesador ARM perfectamente capaz de ejecutar un sistema
operativo completo como GNU/Linux y de soportar el intérprete oficial de Python.

Mientras que este servidor coincide en gran medida en la elección de lenguaje
y bibliotecas utilizadas, su implementación es específica para el robot ``elBrian''
y no puede ser portada a robots con menores capacidades de procesamiento
sin una reescritura significativa. Además el protocolo utilizado no contempla
el acceso a valores de sensores siendo los únicos mensajes que permite enviar
al robot los comandos de movimientos.

Referencias del proyecto:
\begin{itemize}
    \item Página del proyecto: \url{http://www.educabot.org/}
    \item Código fuente de ``elBrian'': \url{https://github.com/educabot/elBraian}
\end{itemize}

\section{Gobot con cppp-io}

Gobot\footnote{\url{http://gobot.io}}
es una biblioteca que permite controlar robots programando en el lenguaje
Go\footnote{\url{https://golang.org/}}. Esta biblioteca soporta el
protocolo Firmata
para controlar robots
conectados directamente a través de una interfaz serial, como es el caso
de los robots Multiplo N6, y soporta la API
cppp-io\footnote{\url{https://github.com/hybridgroup/cppp-io/}}
que define una API JSON
que permite el acceso a la información y control de robots a través de la Web.

Gobot además tiene compatibilidad con distintos sensores y robots, además de
ser compatible con
placas utilizadas normalmente en la construcción de robots como Arduino,
Raspberry Pi, Intel Edison y Beaglebone Black%
\footnote{\url{http://gobot.io/\#platforms}}.

Este proyecto es interesante como base para desarrollar algún proyecto
similar a XRemoteBot en Go, pero requiere además la reimplementación
del módulo de Python DuinoBot que controla, a través de una versión
modificada del protocolo Firmata, a los robots Multiplo N6. Por otro
lado, los robots Scribbler tampoco aparecen en la lista de robots soportados.

Referencias del proyecto:
\begin{itemize}
    \item Página del proyecto: \url{http://gobot.io}.
    \item Protocolo de su API Web: \url{https://github.com/hybridgroup/cppp-io/}.
\end{itemize}


\section{Cylon.js}

Cylon.js es un proyecto hermano de Gobot y soporta la misma variedad
de dispositivos y también provee una API cppp-io, con la diferencia
de que la biblioteca provista está escrita en Javascript para NodeJS.

Una característica interesante de Cylon.js es que soporta distintos
protocolos para su API remota como ser HTTP, socket.io y la capacidad
de agregar nuevos protocolos con
plugins

\begin{lstlisting}[caption={Cylon.js controlando 2 robots ``sphero'' usando HTTP},
label={lst:cylonjs_ejemplo}]
"use strict";

var Cylon = require("cylon");

// ensure you install the API plugin first:
// $ npm install cylon-api-http
Cylon.api({
  host: "0.0.0.0",
  port: "8080"
});

var bots = {
  "Thelma": "/dev/rfcomm0",
  "Louise": "/dev/rfcomm1"
};

Object.keys(bots).forEach(function(name) {
  var port = bots[name];

  Cylon.robot({
    name: name,

    connections: {
      sphero: { adaptor: "sphero", port: port }
    },

    devices: {
      sphero: { driver: "sphero" }
    },

    work: function(my) {
      every((1).seconds(), function() {
        console.log(my.name);
        my.sphero.setRandomColor();
        my.sphero.roll(60, Math.floor(Math.random() * 360));
      });
    }
  });
});

Cylon.start();
\end{lstlisting}

En el código~\ref{lst:cylonjs_ejemplo} se puede ver un ejemplo de
Cylon.js del lado del servidor exponiendo dos robots de tipo
\textit{sphero} a través de la API HTTP,
una vez ejecutado este código en el servidor, a través de la API se
les podrá enviar un mensaje a los robots
robots que hará que cambien
de color
y se desplacen
rodando\footnote{Los robots sphero son pequeñas esferas \url{http://www.gosphero.com/}.}.

Si bien en la documentación del proyecto se detalla como habilitar
un sistema de login para acceder a la API de forma remota, el mismo
no parece estar diseñado para el acceso concurrente de distintos usuarios
ya que de la documentación no surge que Cylon.js soporte un sistema
de reserva de robots o de exclusión mutua en el acceso a los mismos.

Referencias del proyecto:
\begin{itemize}
    \item Página del proyecto: \url{http://cylonjs.com/}.
\end{itemize}


\section{VCar}

\begin{figure}
    \centering
    \includegraphics[width=0.5\linewidth]{figures/vcar}
    \caption{Interfaz de VCar}
    \label{fig:vcar}
\end{figure}

VCar es un auto a control remoto, cuyo control fue modificado para
se manipulado por software desde  una computadora con GNU/Linux la
cuál permite controlar el robot de forma remota y verlo
a través de una cámara de
video\footnote{\url{http://www.hellspark.com/new/css/vcar/vcar.html}}
(figura~\ref{fig:vcar}).
El acceso al robot es público
aparentemente sin ningún tipo de restricción y la página
si bien no tiene detalles de todo el hardware y el software utilizados
provee algunas fotos de la construcción del
sistema\footnote{\url{http://www.hellspark.com/dm/gallery2/v/projects/robotics/vcar/}}

Referencias del proyecto:
\begin{itemize}
    \item Página del proyecto: \url{http://www.hellspark.com/new/css/vcar/vcar.html}.
    \item Galería de fotos: \url{http://www.hellspark.com/dm/gallery2/v/projects/robotics/vcar/}}
\end{itemize}


\section{Tele Toyland}

Tele Toyland\footnote{\url{http://www.teletoyland.com}} es un sitio que provee acceso a varios dispositivos a través de una interfaz
web. Por ejemplo, es posible controlar un cabezal con una punta que dibuja sobre
una caja de arena. Basta con hacer clic sobre las posiciones sobre las cuales
se quiere que pase la punta y presionar el botón ``go'' para que el cabezal
empiece a moverse dibujando lo pedido. En éste y el resto de los experimentos
disponibles en el sitio, los resultados se pueden ver a través de un streaming
de video.

Entre los proyectos con los que permite interactuar Tele Toyland
se encuentran  2 areneros como el descripto
anteriormente, una torre de leds, una marioneta y laberintos.

El sitio no provee detalles del software, ni el protocolo utilizado. Desde
lo funcional
provee una especie de control remoto para los distintos dispositivos a los
que permite
manipular, en donde se puede, incluso, configurar una serie de instrucciones a ejecutar
en secuencia. Sin embargo no provee una biblioteca que permita controlar
los dispositivos desde un lenguaje de programación.

Referencias del proyecto:
\begin{itemize}
    \item Página del proyecto: \url{http://www.teletoyland.com}.
\end{itemize}
% FIXME
%\section{}
%http://www3.uji.es/~pnebot/Files/Articuls/RemoteProgramming.pdf
%Otro
%http://telerobot.mech.uwa.edu.au/Telerobot/instructions.html
\section{Otros proyectos}
% www.linuxuser.co.uk/tutorials/control-your-raspberry-pi-robot-from-a-web-connected-device



%FIX ME ESTO QUEDA MUY DESCOLGADO..... yo pondŕia algo como 
%Bajo la consigna DIY, o ``do it yourself'' y acá poner un par de ref!!!, existen diversas.....y seguir...., 

Existieron otros sistemas similares en el pasado, pero actualmente se
encuentran fuera de línea por problemas técnicos o bien porque el
experimento terminó como por ejemplo
``TU Freiberg Robotics''\footnote{\url{www.informatik.tu-freiberg.de/\~frobots/}}
que permitía controlar robots en una mini cancha de fútbol,
``WebTruck''\footnote{\url{http://www.webtruck.org/cinteraction_html}} que
permitía controlar distintos vehículos.

También existen otros proyectos que requieren instalar software extra
como ``The UWA Telerobot'' que requiere instalar software propietario
y crear una cuenta para ser utilizado, este proyecto permite controlar
un brazo robótico en un taller.

Finalmente en la consigna de ``do it yourself'' existen diversas guías para
programar servidores que permitan controlar robots o microcontroladores
en general, se puede encontrar un caso muy bien explicado en el sitio
de Adafruit~\url{https://learn.adafruit.com/wifi-controlled-mobile-robot/building-the-web-interface},
este es un buen ejercicio de programación, sobre todo para aprender a
programar servidores que provean una API web y clientes que la consuman. Sin
embargo estas guías son introductorias y el objetivo es crear un servidor
muy simple, similar a lo que fue el servidor RemoteBot, pensados para ser
usados en un ambiente local ya que no proveen autenticación en general.

\section{Conclusiones del capítulo}

Si bien existen proyectos parecidos que brindan al público la posibilidad
de controlar un robot de manera remota, ninguno de estos es una solución
completa al problema de enseñar a programar usando
robots de forma remota y permitiendo a distintos usuarios reservar
temporalmente distintos
robots de forma que no haya conflicto en la interacción entre los usuarios.
