% El capítulo 4 describe la implementación de los clientes y el modo de uso
% de cada uno destacando algunas diferencias y decisiones de diseño en la
% implementación en Javascript.

\chapter{Implementación de los clientes}\label{cha:clientes}

La implementación de XRemoteBot para esta tesina incluye tres clientes en distintos
lenguajes de programación: Python, Ruby y Javascript. Este último para ejecutarse
en el entorno de un navegador Web.

Estos clientes tienen distintas características y principalmente el que más difiere
de los otros es el cliente Javascript ya que es una implementación a ser ejecutada
en el navegador con las restricciones que esto trae, pero con la ventaja de que
se encuentra integrado con la vista Web de XRemoteBot por lo que cuenta con una
vista de la cámara al lado del campo de texto donde el usuario puede codificar.

En general se intentó copiar el estilo de programación del módulo DuinoBot original
de la forma más fiel posible, en el caso de Javascript por limitaciones del entorno
no se logró imitar este estilo tan fielmente como en los otros clientes.

\section{Cliente Python}\label{sec:python}
El uso del cliente Python es muy similar al uso directo de los robots con la
biblioteca DuinoBot aunque la implementación es bastante diferente por el protocolo
subyacente.

El cliente Python utiliza un solo módulo que no pertenece a la biblioteca estándar
de Python, el módulo para acceso a WebSockets
\textit{websocket-client}\footnote{\url{https://pypi.python.org/pypi/websocket-client}}
que cuenta, además de con una interfaz asincrónica, con una interfaz de ``bajo nivel''
sincrónica similar a la interfaz tradicional de Sockets.

\begin{lstlisting}[language=Python,
caption={Ejemplo con XRemoteBot para Python},label=lst:ejemplo_xremotebot_python]
from xremotebot import *
server = Server('ws://xremotebot.example:8000/api', 'api_key')
robot = Robot(server, server.fetch_robot())
robot.forward(100, 1)
robot.backward(50, 2)
print(robot.getObstacle())
\end{lstlisting}


Las operaciones con demoras, como los movimientos se implementan localmente
utilizando 2 mensajes y la función \texttt{time.wait()} de Python. De esta
forma
invocar al método \texttt{robot.forward(100, 1)} se traduce en los mensajes
de
la figura~\ref{fig:ejemplo_xremotebot_python_json}. Como todos los métodos
de movimiento
proveen esta funcionalidad de demora, la misma implementada en el
\textit{decorator}\footnote{Esto es un decorator de Python, no confundir
con el patrón de diseño. \url{https://www.python.org/dev/peps/pep-0318/}}
\texttt{xremotebot.timed} a fin de evitar repetición de código.

\begin{lstlisting}[language=Python,
caption={Mensajes generados al invocar \texttt{robot.forward(100, 1)} en
XRemoteBot para Python}, label=lst:ejemplo_xremotebot_python_json]
{
    "entity": "robot",
    "method": "forward",
    "args": [100]
}
# Después de 1 segundo...
{
    "entity": "robot",
    "method": "stop",
    "args": []
}
\end{lstlisting}


\section{Cliente Ruby}\label{ch4:ruby}

Las gemas que proveen soporte para WebSockets en Ruby proveen una API
asincrónica que no es deseable para este proyecto por lo que me incliné
por incorporar el código del
proyecto\footnote{\url{https://github.com/gimite/web-socket-ruby}}
al cliente Ruby de XRemoteBot, este proyecto no se encuentra empaquetado
en forma de gema pero provee una API sincrónica que permite implementar
el cliente Ruby de XRemoteBot para que se comporte de una forma muy similar
a la biblioteca DuinoBot original.

\begin{lstlisting}[language=Ruby,
caption={Ejemplo usando XRemoteBot para Ruby},
label=lst:ejemplo_xremotebot_ruby]
require 'xremotebot'
server = XRemoteBot::Server.new('xremotebot.example',
                                8000,
                                'api',
                                'api_key')

print "Robots #{server.get_robots}\n"
                                                                                                                    robot = Robot.new server, server.fetch_robot
robot.forward 100, 1
robot.backward 50, 2
print robot.getObstacle
\end{lstlisting}



\section{Cliente Javascript}\label{ch4:javascript}

\subsection{API Javascript y asincronismo}
Como se mencionó con anterioridad, gran parte de las decisiones de diseño de XRemoteBot
se hicieron para permitir la implementación de un cliente Javascript, pero dado el hecho
de que Javascript dentro del entorno de un navegador Web se comporta de forma asincrónica
no fue posible implementar un cliente Javascript cuyo uso se asemeje al de la biblioteca
de Python DuinoBot.

Para ilustrar esta problemática se puede tomar en consideración el 
código~\ref{lst:ejemplo_duinobot}
típico hecho usando DuinoBot en Python y analizar los inconvenientes para replicar
algo similar en Javascript.

\begin{lstlisting}[language=Python,
caption={Ejemplo típico usando DuinoBot},label=lst:ejemplo_duinobot]
from duinobot import *
board = Board()
robot = Robot(board, 10)
robot.forward(100, 1)
robot.backward(50, 2)
print(robot.getObstacle())
\end{lstlisting}


Las líneas 2 y 3 del ejemplo de la código~\ref{lst:ejemplo_duinobot} son
operaciones que retornan objetos, en XRemoteBot la línea 2 no tiene sentido
ya que la placa a la que está conectado el robot está predefinida en el
servidor, pero la línea 3 sí es necesaria para obtener una instancia de un
robot no reservado, estas operaciones que retornan un valor el XRemoteBot
se dividen en un mensaje de petición y uno de respuesta, el problema
es que por el asincronismo de Javascript y de la API de WebSockets no
es posible determinar en que momento llegará la respuesta con el valor
retornado y la única forma de bloquear la ejecución del script hasta que
llegue la respuesta sería hacer ``busy waiting'', lo cuál puede llegar a
bloquear la pestaña del navegador ya que los navegadores típicamente usan
un solo thread compartido entre el renderizado de la página, manejo de
eventos y ejecución del código Javascript asociado a la página.

En tanto las líneas 4 y 5 si bien no retornan un valor deben ejecutarse
en orden, y la línea 5 en este caso debe ejecutarse precisamente 1 segundo
después de la ejecución de la línea 4, también la línea 6 deberá ejecutarse
2 segundos después de la línea 5 ya que no es lo mismo preguntar si hay
un obstáculo cuando el robot aún no retrocedió, esto que parece muy obvio
y natural en la mayoría de los lenguajes de programación no es posible
en Javascript (dentro del navegador, con intérpretes como NodeJS sí es
posible) y el orden de ejecución y demora no se puede lograr simplemente
poniendo una instrucción debajo de otra.\footnote{FIXME}


La solución encontrada a este problema fue el uso de
\textit{Promises}~\citep{ECMA-262},
las \textit{Promises} proveen un mecanismo para ejecutar instrucciones,
que dependen de los resultados o de la terminación de eventos asincrónicos,
las \textit{Promises} permiten tener secuencia e interdependencia entre los
mensajes enviados a los
robots.
Si bien el objeto \textit{Promise} está en proceso de estandarización,
los navegadores Google Chrome, Firefox, Internet Explorer, Opera y Safari las
soportan\footnote{\url{https://developer.mozilla.org/en-US/docs/Web/JavaScript/Reference/Global_Objects/Promise}}.
El problema de esta solución es que la sintaxis de un script XRemoteBot en
Javascript usando \textit{Promises}
resulta bastante distinta de la sintaxis de un script DuinoBot como resulta
evidente
en el código~\ref{lst:ejemplo_xremotebot_javascript} donde se muestra un
script equivalente
al del código~\ref{lst:ejemplo_duinobot} escrito en Javascript.

\begin{lstlisting}[language=C,
caption={Ejemplo de XRemoteBot en Javascript},
label=lst:ejemplo_xremotebot_javascript]
var server = new Server('xremotebot.example:8000', 'api-key');
server.onConnect(function(){
    server.fetch_robot().then(function(robot){
        robot.forward(100, 1).then(function(){
            robot.backward(50, 2).then(function(){
                robot.getObstacle().then(function(obstacle){
                    println(obstacle);
                })
            })
        })
    });
});
\end{lstlisting}

Como se puede ver después de cada acción bloqueante, es decir cada método que
provoque una demora, y de cada método que retorne un valor útil, como
\texttt{Robot\#getObstacle()}, se invoca el método \texttt{Promise\#then()} con
una función anónima como argumento. Esta función que se pasa por argumento al
then se ejecutará si la \textit{Promise} se ``resuelve'', en el caso de
XRemoteBot la \textit{Promise} se ``resuelve'' cuando el servidor
responde al mensaje que generó la \textit{Promise}.
Este mecanismo se logró identificando cada mensaje con un \texttt{msg\_id},
manteniendo un objeto Javascript (usado como si fuera un \texttt{dict}
o \texttt{HashMap}) con el identificador como clave y la promesa
correspondiente a cada uno los mensajes cuyas respuestas están pendientes
como valor. Dentro del callback \texttt{onmessage} del WebSocket utilizado
al recibir una respuesta se busca el \texttt{msg\_id} en el objeto de
mensajes con respuesta pendiente y se ``resuelve'' la promesa correspondiente
a este mensaje. De esta manera se ejecuta la siguiente función (pasada como
argumento a \texttt{Promise\#then}) siguiendo con el flujo del programa
en orden y con la demora necesaria. En el
código~\ref{lst:ejemplo_xremotebot_javascript_promises} se muestra esta primer
implementación.


\begin{lstlisting}[language=C,
caption={Ejemplo simplificado de la implementación de XRemoteBot con
Promises dentro del constructor Server en remotebot.js},
label=lst:ejemplo_xremotebot_javascript_promises]
this.send_ws_msg = function(msg){
    var promise;
    promise = new Promise(function(resolve, reject){
        that.pending_msgs[msg_id] = {resolve: resolve, reject: reject};
    });
    msg['msg_id'] = msg_id;
    that.ws.send(JSON.stringify(msg));
    return promise;
}
// ...
this.ws.onmessage = function(msg){
    msg = JSON.parse(msg.data);
    if (msg['msg_id'] !== undefined){
        var executor = that.pending_msgs[msg['msg_id']];
        delete that.pending_msgs[msg['msg_id']];
        executor.resolve(msg);
    }
};
\end{lstlisting}


Evidentemente esta forma de programar resulta engorrosa y poco práctica,
por lo que se añadieron
lógica y estructuras de datos extra que serializan en envío de mensajes al
servidor, enviando de a un mensaje por vez (el objeto Javascript con
mensajes con respuesta pendiente tendrá a lo sumo una entrada)
y encolando los mensajes sobrantes
en una cola de mensajes demorados (\texttt{Server\#delayed}), cuando
el servidor contesta el mensaje enviado anteriormente se toma un mensaje
de la cola si lo hubiere y se lo envía al servidor. De esta manera
se garantiza la demora necesaria entre la ejecución de cada método
del robot, sin embargo para obtener los valores de retorno de los métodos,
como en el caso de los métodos de acceso a los sensores, sigue siendo
necesario el uso del método \texttt{Promise\#then}. A pesar de esto último
como se puede ver en el código~\ref{lst:ejemplo_xremotebot_javascript_cola}
con estas modificaciones es posible hacer que el cliente Javascript
tenga una API más limpia y usable, esta última versión es la definitiva
para este trabajo.

\begin{lstlisting}[language=C,
caption={Ejemplo de XRemoteBot en Javascript con empleo de una cola para
serializar mensajes},
label=lst:ejemplo_xremotebot_javascript_cola]
var server = new Server('xremotebot.example:8000', 'api-key');
server.fetch_robot().then(function(robot){
    robot.forward(100, 1);
    robot.backward(50, 2);
    robot.getObstacle().then(function(obstacle){
        println(obstacle);
    });
});
\end{lstlisting}


Una solución alternativa para lograr imitar tanto como sea posible la
API de DuinoBot puede ser utilizar un intérprete Javascript implementado en
Javascript como
JS-Interpreter\footnote{\url{https://github.com/NeilFraser/JS-Interpreter}},
pero requeriría modificar el intérprete para lograr una ejecución paso a paso
controlada por el flujo de mensajes a través de la conexión con WebSockets,
además de agregar complejidad, este tipo de intérpretes no accede a un entorno
compreto como lo hace el intérprete incorporado en los navegadores y puede
tener incompatibilidades o funcionalidades no implementadas, en este caso
por ejemplo JS-Interpreter no soporta interrupciones y no puede interactuar
directamente con DOM, otra opción es
Hypnotic\footnote{\url{http://coolwanglu.github.io/hypnotic/web/demo.html}}
que utiliza el intérprete Narcissus y provee una función \texttt{sleep()} que
demora la ejecución del código como se desea para este proyecto, pero el
invoveniente de Hypnotic es que por el momento solamente funciona en el
navegador
Firefox\footnote{\url{https://github.com/coolwanglu/hypnotic/wiki\#limitations}}.


\subsection{Interacción con DOM y JQuery}

\subsection{Interfaz Web y streaming de video}

La interfaz Web de XRemoteBot, que está pensada principalmente para los casos
de uso donde los clientes están en una ubicación geografica distinta a la de
los robots provee login, acceso a la visualización y renovación de una
clave alfanumérica única por cada usuario denominada \textit{API key} que
permite reservar y controlar los robots sin la necesidad de exponer el nombre
de usuario y contraseña, y una página que permite ver los robots en vivo por
video opcionalmente controlandolos con un script Javascript.

Login y renovación de API, doc

La interfaz web pensada para ser utilizada en conjunto con el cliente
Javascript (aunque puede ser accedida sin problemas aún si se usa
otro de los clientes) provee un area de texto para escribir el script
a ejecutar % FIXME
para la misma se utilizó CodeMirror ...

Un area de texto que simula la salida de una terminal, en la misma
se pueden ver mensajes de log generados con la función
\texttt{rblog} que pueden ser habilitados o deshabilitados desde un
checkbox y mensajes impresos con la función \texttt{rbprintln}.

Un area de video destinada a emisiones en vivo que muestren la posición
del robot, el mismo no requiere ningún plugin ya que se utilizó
\texttt{jsmpeg}\footnote{\url{https://github.com/phoboslab/jsmpeg}}
que permite emitir video en vivo usando WebSockets y renderizarlo
en el navegador en un elemento Canvas, de esta manera se logró tener
video en vivo utilizando características estándar de HTML5.
