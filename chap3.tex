%El capítulo 3 describe el protocolo de capa de aplicación diseñado para
%este servidor y las modalidades de operación del servidor.
\chapter{Protocolo de capa de aplicación}\label{ch3}

Como se describió anteriormente XRemoteBot utiliza el protocolo WebSocket como
protocolo de la capa de transporte y mensajes codificados con JSON como
protocolo de la capa de aplicación, el propósito de este capítulo es describir
este protocolo de capa de aplicación y detallar los motivos por los cuales
se modeló el protocolo de esta manera.


El protocolo fue diseñado pensando que el mismo debería representar los
distintos comandos envíados a los robots como si fueran mensajes a objetos
del paradigma de programación orientado a objetos
este diseño por un lado mapea de forma muy directa los mensajes de la API
con los métodos que un programador utilizaría para controlar a los robots
usando la biblioteca DuinoBot y por otro lado provee un alto grado de
flexibilidad a la hora de agregar nuevos tipos de mensajes a la API sin
necesidad de reescribir grandes porciones de código.

Con la misma idea de hacer un mapeo directo entre los mensajes de la API
y la idea de mensaje en el paradigma orientado a objetos, luego de ejecutar
el método asociado con el mensaje recibido, en los lenguajes orientados a
objetos más populares un objeto puede retornar un valor de respuesta o fallar
con una excepción, por esto el servidor cuenta con 2 tipos de respuesta
``value'' que retorna el valor de respuesta al mensaje recibido y
``error'' que representa un error que típicamente se representaría como
una excepción contando con un mensaje de error descriptivo.

Tanto para los parámetros, como para los valores de respuesta, los tipos
de datos representables de forma nativa con JSON resultan suficientes para
implementar una API similar a la de DuinoBot. Internamente XRemoteBot
convierte los valores, objetos y arrays de JSON a distintos tipos de
datos de Python de la forma descripta en la
tabla~\ref{tbl:rel_json_python}, siguiendo en general las transformaciones
realizadas por defecto por la libería \texttt{json} de
Python~\footnote{\url{https://docs.python.org/3/library/json.html}}
excepto en algunos mensajes donde los valores de
tipo \texttt{number} de JSON se convierten explícitamente.

\begin{table}
    \centering
    \begin{tabu}{l|l}
        JSON & Python \\
        \hline
        \texttt{string} & \texttt{str} \\
        \texttt{number} & \texttt{int} o \texttt{float}~\footnote{Dependiendo
        del mensaje invocado
        puede ser interpretado como \texttt{int} o \texttt{float}.}\\
        \texttt{object} & \texttt{dict} \\
        \texttt{array}  & \texttt{list} \\
        \texttt{true}/\texttt{false} & \texttt{True}/\texttt{False} (bool) \\
        \texttt{null} & \texttt{None} \\
    \end{tabu}
    \caption{Relación entre los tipos y constantes de JSON y los usados en
    Python}
    \label{tbl:rel_json_python}
\end{table}

\section{Mensajes del cliente al servidor}

Los mensajes al servidor son objetos JSON con dos campos obligatorios y dos
opcionales, estos campos son:

\begin{description}
    \item[\texttt{entity}] su valor es un string que describe conceptualmente
        el receptor del mensaje (obligatorio).
    \item[\texttt{method}] su valor es un string con el nombre del método a
        invocar (obligatorio).
    \item[\texttt{args}] su valor es un array posiblemente vacío con los
        argumentos que deben pasarse al método invocado (opcional).
    \item[\texttt{msg\_id}] si está presente el servidor incorpora este
        campo con su valor original en la respuesta, sirve para identificar a
        qué petición
        corresponde cada respuesta en implementaciones asincrónicas
        como la de Javascript.
\end{description}

En la actual implementación los valores posibles de \texttt{entity} son
\texttt{``global''} y \texttt{``robot''}, el primero
representa acciones globales como el login y el último representa una
abstracción de los robots soportados por XRemoteBot.

La entidad ``global'' soporta los métodos:
\begin{description}
    \item[\texttt{authentication\_required}] no recibe argumentos, retorna
        \texttt{true} si el servidor está configurado para requerir
        autenticación y \texttt{false} en caso contrario.
    \item[\texttt{authenticate}] recibe un string, si el string se corresponde
        con la \texttt{api\_key} de algún usuario y esa \texttt{api\_key} no
        ha expirado retorna \texttt{true} y \texttt{false} en caso contrario.
\end{description}

Entre otros la entidad ``robot'' soporta los métodos:
\begin{description}
    \item[\texttt{backward}] recibe un objeto que identifica a un robot
        específico, una velocidad y un tiempo en segundos,
        mueve el robot hacia atrás el tiempo indicado y luego contesta
        al cliente. Si no se envía el parámetro de tiempo el robot se mueve
        de forma indefinida.
    \item[\texttt{forward}] ídem moviendo el robot hacia adelante.
    \item[\texttt{turnLeft}] ídem girando a izquierda.
    \item[\texttt{turnRight}] ídem girando a derecha.
    \item[\texttt{getObstacle}] recibe un objeto que identifica a un robot,
        opcionalmente una distancia y retorna
        \texttt{true} si hay un obstáculo a una distancia menor o igual a la
        indicada, si no se envía la distancia se asume un valor por defecto.
    \item[\texttt{getLine}] recibe un objeto que identifica a un robot y
        retorna un \texttt{array}
        con los valores de los sensores de línea del robot.
\end{description}

Opcionalmente un modelo dado de robot puede soportar otros métodos o retornar
error en alguno de los métodos anteriores.

La posiblidad que tienen los métodos con tiempo como \texttt{backward} de
contestar la petición varios segundos después de recibirla sin afectar
el funcionamiento del servidor ni demorar innesesariamente al resto de los
clientes está dada por el soporte de asincronismo del framework Tornado.

\section{Mensajes del servidor a los clientes}

El servidor responde a los clientes con mensajes que contienen una
entrada \texttt{response}, esta entrada identifica el tipo de respuesta
que puede ser \texttt{value} o \texttt{error}, si el cliente envío un
campo \texttt{msg\_id} en la petición el servidor incorporará además
un campo \texttt{msg\_id} idéntico en la respuesta correspondiente
a esa petición (como se mencionó anteriormente para soportar
clientes asincrónicos de forma correcta).

Campos de los mensajes \texttt{value}:
\begin{description}
    \item[\texttt{response}] el string \texttt{``value''}.
    \item[\texttt{value}] el valor retornado por el método (puede ser
        cualquier valor soportado por JSON).
    \item[\texttt{msg\_id}] si la petición tenía \texttt{msg\_id} se
        copia el mismo, sino este campo se omite.
\end{description}

Campos de los mensajes \texttt{error}:
\begin{description}
    \item[\texttt{response}] el string \texttt{``error''}.
    \item[\texttt{error}] un mensaje de error descriptivo.
    \item[\texttt{msg\_id}] si la petición tenía \texttt{msg\_id} se
        copia el mismo, sino este campo se omite.
\end{description}

Los mensajes \texttt{value} representan los valores de respuesta de
los métodos invocados, para los mensajes que requieren
un tiempo de demora el servidor contestará con el mensaje \texttt{value}
una vez transcurrido ese tiempo, para no demorar la atención de peticiones
de otros clientes el servidor utiliza las funcionalidades provistas por
Tornado para atender peticiones de forma asíncrona~\citep{dory_2012}. Por
otro lado los mensajes \texttt{error} representan eventos anómalos que
generalmente en un sistema no distribuido se modelarían utilizando
excepciones, por ejemplo errores de codificación, peticiones a recursos
ocupados, etc...

\section{Ejemplos de interacción entre los clientes y el servidor}

Al comenzar la conexión entre el cliente y el servidor, se intercambian
los primeros mensajes donde el cliente pregunta al servidor si el
mismo requiere autenticación y si es así envía la \textit{API key}
correspondiente. En la tabla~\ref{tbl:ej_autenticacion} se detalla
el intercambio de mensajes entre un cliente y el servidor en un
caso de autenticación exitoso con un cliente que envía un \texttt{msg\_id},
como se puede ver el \texttt{msg\_id}
no tiene por qué ser consecutivo con el del mensaje anterior, incluso
puede llegar a ser un string, las únicas restricciones sobre el
\texttt{msg\_id} son que tiene que ser copiado por el servidor
en las respuestas y que el valor debe poder ser utilizado como
identificador de un atributo (usando notación con corchetes) en un
objeto Javascript, es decir puede ser un número o un string
arbitrario~\footnote{\url{https://developer.mozilla.org/en-US/docs/Web/JavaScript/Reference/Operators/Property\_Accessors}}.

\begin{table}
    \centering
    \begin{tabular}{|m{.49\textwidth}|m{.49\textwidth}|}
        \hline
        \textit{Petición del cliente} & \textit{Respuesta del servidor} \\
        \hline
\begin{Verbatim}[fontsize=\footnotesize]
{
    "entity": "global",
    "method": "authentication_required",
    "msg_id": 0
}
\end{Verbatim}
&
\begin{Verbatim}[fontsize=\footnotesize]
{
    "response": "value",
    "value": true,
    "msg_id": 0
}
\end{Verbatim}
\\
\hline
\begin{Verbatim}[fontsize=\footnotesize]
{
    "entity": "global",
    "method": "authenticate",
    "args": ["16758efc752c1d97381e6"],
    "msg_id": 20
}
\end{Verbatim}
&
\begin{Verbatim}[fontsize=\footnotesize]
{
    "response": "value",
    "value": true,
    "msg_id": 20
}
\end{Verbatim}
\\
\hline
    \end{tabular}
    \caption{Ejemplo de secuencia de autenticación}
    \label{tbl:ej_autenticacion}
\end{table}

Otra secuencia de operaciones típica para XRemoteBot es mover un robot
y luego pedir los valores de alguno de sus sensores, en la
tabla~\ref{tbl:ej_movimiento_y_sensor} se puede ver un caso donde se
mueve el robot hacia adelante durante 1 segundo y luego consulta si
existe algún obstáculo a 10 centímetros o menos del robot, en el primer
argumento de los mensajes
envíados a la entidad \textit{robot} se puede ver el objeto que identifica
al robot específico a utilizar.

\begin{table}
    \centering
    \begin{tabular}{|m{.49\textwidth}|m{.49\textwidth}|}
        \hline
        \textit{Petición del cliente} & \textit{Respuesta del servidor} \\
        \hline
\begin{Verbatim}[fontsize=\footnotesize]
{
    "entity": "robot",
    "method": "forward",
    "args": [50, 1],
    "msg_id": 40
}
\end{Verbatim}
&
\begin{Verbatim}[fontsize=\footnotesize]
{
    "response": "value",
    "value": null,
    "msg_id": 40
}
\end{Verbatim}
\\
\hline
\begin{Verbatim}[fontsize=\footnotesize]
{
    "entity": "robot",
    "method": "getObstacle",
    "args": [{model: "n6", id: 4}, 10],
    "msg_id": 42
}
\end{Verbatim}
&
\begin{Verbatim}[fontsize=\footnotesize]
{
    "response": "value",
    "value": false,
    "msg_id": 42
}
\end{Verbatim}
\\
\hline
    \end{tabular}
    \caption{Ejemplo de movimiento y acceso a sensores de un robot}
    \label{tbl:ej_movimiento_y_sensor}
\end{table}

Por último en la tabla~\ref{tbl:ej_error} se puede ver un mensaje de error
típico cuando el cliente envía una petición mal formada donde no existe
el campo \textit{method} obligatorio, notar que este mensaje a diferencia
de los anteriores no tiene un campo \textit{msg\_id} pero esto no causaría
ningún error ya que este campo no es obligatorio, el único error en la
petición es la ausencia de \textit{method}.

\begin{table}
    \centering
    \begin{tabular}{|m{.49\textwidth}|m{.49\textwidth}|}
        \hline
        \textit{Petición del cliente} & \textit{Respuesta del servidor} \\
        \hline
\begin{Verbatim}[fontsize=\footnotesize]
{
    "entity": "robot",
}
\end{Verbatim}
&
\begin{Verbatim}[fontsize=\footnotesize]
{
    "response": "error",
    "message": "\"entity\" and \"method\"
                are mandatory fields"
}
\end{Verbatim}
\\
\hline
    \end{tabular}
    \caption{Ejemplo de un mensaje de error ante una petición mal formada}
    \label{tbl:ej_error}
\end{table}

\section{Modalidades del servidor}

A fin de hacer que el servidor pueda exponerse al público se cuenta con
un sistema con autenticación por \textit{API key}, pero estas claves
son largas resultando díficil recordarlas y escribirlas sin errores. En
ámbitos locales como puede ser una red wifi en un aula este mecanismo
de autenticación puede resultar molesto e innecesario, por lo que el
servidor es configurable de forma tal que se puede deshabilitar este
sistema de autenticación. Para que el servidor opere sin requerir
una \textit{API key} a los clientes basta con configurar la opción
\texttt{public\_server} en \texttt{False} dentro del archivo
\textit{configuration.py}.


