% El capítulo 4 describe la implementación de los clientes y el modo de uso
% de cada uno destacando algunas diferencias y decisiones de diseño en la
% implementación en Javascript.

\chapter{Implementación de los clientes}\label{ch4}

\section{Cliente Python}\label{ch4:python}
\section{Cliente Ruby}\label{ch4:ruby}
\section{Cliente Javascript}\label{ch4:javascript}

Como se mencionó con anterioridad, gran parte de las decisiones de diseño de XRemoteBot
se hicieron para permitir la implementación de un cliente Javascript, pero dado el hecho
de que Javascript dentro del entorno de un navegador Web se comporta de forma asincrónica
no fue posible implementar un cliente Javascript cuyo uso se asemeje al de la biblioteca
de Python DuinoBot.

Para ilustrar esta problemática se puede tomar en consideración el script de la
figura~\ref{lst:duinobot_ejemplo}
típico hecho usando DuinoBot en Python y analizar los inconvenientes para replicar
algo similar en Javascript.

\begin{figure}
    \begin{lstlisting}[language=Python,numbers=left]
from duinobot import *
b = Board()
r = Robot(b, 10)
r.forward(100, 1)
r.backward(50, 2)
print(r.getObstacle())
    \end{lstlisting}
    \caption{Ejemplo típico usando DuinoBot}
    \label{lst:duinobot_ejemplo}
\end{figure}
