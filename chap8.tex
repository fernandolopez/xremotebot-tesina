\chapter{Conclusiones y trabajo a futuro}\label{cha:conclusiones}
Esta tesina permitió desarrollar una aplicación web basada en tecnologías
abiertas, llamada XRemoteBot, que permite manipular y visualizar en tiempo
real los robots Multiplo N6 de la empresa RobotGroup y que se aplicará al
proyecto \proyecto{}. Se implementaron tres clientes, para tres lenguajes de
programación: Python, Ruby y Javascript, que permiten ampliar los alcances
del proyecto, por un lado incorporando nuevos lenguajes de programación para
programar los robots y por otro lado, facilitando la incorporación de nuevos
destinatarios del proyecto, entre ellos escuelas que no disponen de los
robots Multiplo N6 para el desarrollo de su práctica y aquellas que aún
contando con los robots podrían realizar actividades en forma remota.

XRemoteBot está desarrollado en Python y utiliza las siguientes tecnologías:
\begin{itemize}
    \item Tornado como framework web y soporte de WebSockets del lado del
    servidor.
    \item SQLAlchemy para el acceso a la base de datos.
    \item DuinoBot para controlar los robots ``N6''.
    \item Myro para controlar los robots ``Scribbler''.
    \item Sniffer\footnote{\url{https://pypi.python.org/pypi/sniffer}} y
    Nose\footnote{\url{https://nose.readthedocs.org/en/latest/}} para correr
    los tests de forma automatizada.
    \item Diversos módulos de la biblioteca estándar.
\end{itemize}

El framework Tornado resultó ser adecuado para el uso de WebSockets y el
manejo de demoras en las respuestas debido al soporte incorporado para
WebSockets; a su vez cuenta con diversas utilidades para la atención de
peticiones de forma asincrónica.

El protocolo WebSocket fue
elegido ya que está estandarizado y su uso se está popularizando cada vez más,
y si bien la API WebSockets está en proceso de estandarización es soportada
por los navegadores más usados en la actualidad.

Para disminuir la cantidad de bytes de información enviados sin impactar en el
uso de CPU tanto del servidor como de los clientes, se hizo un estudio con
múltiples pruebas en Python y Javascript con los mecanismos de serialización
JSON, CBOR y BSON, el primero de éstos de texto plano y soportado de forma
nativa en los tres clientes implementados y los restantes mecanismos son
binarios. Estas pruebas, presentadas en la sección~\ref{sec:serializacion},
dieron como resultado que el uso de JSON no genera mensajes significativamente
más grandes que las otras alternativas analizadas (incluso son ligeramente más
pequeños que los generados con BSON), a la vez que el tiempo para codificar y
decodificar los mensajes con JSON en Javascript es menor que las otras
alternativas y en Python no presenta
diferencias significativas para mensajes pequeños como los utilizado
por XRemoteBot.

La API de XRemoteBot fue probada con 3 clientes distintos incorporados en este
trabajo, de los cuales el cliente Javascript se puede ejecutar desde el
navegador (inclusive desde dispositivos móviles) y los clientes Python y
Ruby se pueden usar en cualquier computadora
donde se puedan instalar sus intérpretes (abarcando computadoras con
los sistemas operativos GNU/Linux, *BSD, Windows y Mac).

Los clientes Ruby y Python son los más simples y es posible extenderlos
fácilmente para agregarles nuevas funcionalidades.

El sistema resultante es fácilmente extensible en tres aspectos:
\begin{itemize}
    \item Dada su sencillez el protocolo es fácilmente extensible.
    \item Definiendo un script en ``remotebot/robots/<nombre>.py''
en el servidor y agregando ``<nombre>'' en la opción ``robots''
de la configuración es suficiente para agregar soporte
para un nuevo robot.
    \item Si se requiere controlar un dispositivo tan distinto de los
        robots que requeriría métodos muy distintos se pueden definir
        nuevas entidades, una entidad simplemente es una clase que
        hereda de ``remotebot.models.entity.Entity'' y que se
        registra al final del archivo ``remotebot/handlers/ws\_handler.py''
        con el método \texttt{WSHandler.register\_api\_handler()}.
        En el método anterior recibe como primer argumento el
        nombre de la entidad y como segundo argumento una instancia
        de una clase que herede de ``remotebot.models.entity.Entity''.
\end{itemize}

XRemoteBot contribuirá como dispositivo educativo a una de las líneas
de investigación del LINTI vinculada con acercar la programación a la
escuela. Específicamente se utilizará en el proyecto \proyecto{} brindando
la posibilidad de contar con una herramienta tecnológica que permita
trabajar en forma remota, ofreciendo la oportunidad de trabajar con escuelas
de puntos geográficos alejados y con aquellas que no cuenten con robots
reales.

Todos los desarrollos implementados se han publicado con una licencia
MIT y pueden ser descargados desde el sitio
GitHub\footnote{\url{http://github.com/fernandolopez/xremotebot} y\\
\url{http://github.com/fernandolopez/xremotebot-clients}}.

\section{Trabajo a futuro}
Habiendo alcanzado la etapa final del desarrollo de XRemoteBot
se plantean una serie de
posibles trabajos derivados y extensiones que permitirían adaptar este
sistema para nuevos requerimientos:

\begin{itemize}
    \item Extender el soporte de lenguajes de programación en la interfaz
        agregando, por ejemplo, soporte a Python y Ruby con Brython u Opal.
    \item Analizar la posibilidad de añadir soporte a la programación con
        visual con la biblioteca
        \texttt{Blockly}\footnote{\url{https://developers.google.com/blockly/}}.
    \item Analizar la posibilidad de integrar la interfaz web de XRemoteBot con
        sistemas de gestión de contenido para educación como
        Moodle\footnote{\url{http://moodle.org}}.
    \item Analizar tecnologías alternativas para el streaming de video en
        redes con anchos de banda relativamente bajos.
\end{itemize}



