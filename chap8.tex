\chapter{Pruebas}\label{cha:pruebas}

El servidor se probó con Python 2.7 (portarlo a Python 3 involucraría
portar DuinoBot, Myro y otros módulos asociados), en entornos creados
con virtualenv\footnote{FIXME REF} y con una base de datos
Sqlite\footnote{FIXME REF}
sobre
sistemas Lihuen 5 y 6 beta (basados en Debian Wheezy y Jessie respectivamente).
Las dependencias se instalaron declarándolas en \texttt{setup.py} e instalandolas
con pip, a excepción de los módulos DuinoBot y Myro que fueron instalados
desde repositorios Git usando también pip, pero indicando la URL manualmente
ya que no fué posible declararlos de esta manera en \texttt{setup.py}, el
código~\ref{lst:instalar_xremotebot} muestra como instalar el servidor
y los paquetes necesarios para soportar robots Multiplo N6 y Scribbler.

\begin{lstlisting}[language=bash,
caption={Instalación de XRemoteBot},
label=lst:instalar_xremotebot]
git clone https://github.com/fernandolopez/xremotebot.git
cd xremotebot
virtualenv .
. bin/activate
pip install -r requirements.txt
pip install 'git+https://github.com/Robots-Linti/duinobot.git@pygame_opcional'
pip install 'git+https://github.com/fernandolopez/Myro.git'
\end{lstlisting}

\begin{lstlisting}[language=bash,
caption={Instalación del soporte de video para XRemoteBot},
label=lst:instalar_video]
apt-get install libav-tools nodejs
npm install -g ws
\end{lstlisting}

Para soportar el streaming de video además es necesario instalar
\texttt{avconv}, \texttt{NodeJS} y el paquete \texttt{ws} para NodeJS,
se pueden instalarde distintas formas dependiendo del sistema operativo
utilizado, en código~\ref{lst:instalar_video} muestra como instalarlo
en un sistema basado en Debian GNU/Linux.

Finalmente para ejecutar todo es conveniente utilizar el script
\texttt{run} provisto con el servidor que se encarga de ejecutar
avconv, el servidor de jsmpeg y de asociar las MACs de los
robots Scribbler que hubiera configurados con dispositivos
\texttt{rfcomm} de manera que queden utilizables para \texttt{Myro}.
Luego de ejecutar esas tareas el script \texttt{run} ejecuta
el servidor XRemoteBot.

\begin{lstlisting}[language=bash,
caption={Ejecutar el servidor y procesos asociados},
label=lst:ejecutar_xremotebot]
./run.sh
\end{lstlisting}

Si hubiese algún robot Scribbler en la configuración el script
\texttt{run} pedirá la contraseña de administrador para asociar
su MAC con un dispositivo \texttt{rfcomm}.


usando los archivos requirementes.txt
y requirements-dev.txt, a excepción de los módulos DuinoBot y Myro que se
instalaron con pip desde Git por lo que no fue posible usar los archivos

\section{Pruebas de uso de recursos}
Se dejó el servidor ejecutándose durante aproximadamente 24hs en
un equipo con un procesador
``AMD Athlon(tm) 64 X2 Dual Core Processor 4600+'' y 2GB de memoria
RAM. Durante ese tiempo hubo un cliente conectado controlando
un robot (que recibió energía de una fuente ATX para evitar problemas
con las pilas) realizando movimientos cada determinada cantidad de
segundos. Esporádicamente se verificó con la herramienta
\texttt{htop} el uso de recursos general de XRemoteBot y herramientas
asociadas. En estas pruebas ni XRemoteBot ni el servidor de jsmpeg
mostraron valores significativos en el uso de CPU y memoria en
el sistema, en cambio la herramienta \texttt{avconv} consumió
de manera prácticamente constante un 20\% de un núcleo de la CPU
y un 0.3\% de memoria. Al ejecutar un cliente en este mismo
equipo en el navegador Mozilla Firefox 40.0a1 (Nightly)
el uso de CPU por parte de este navegador estuvo alrededor
del 25\%, un 40\% en Google Chrome 41.0.2272.101 (Stable).

Eliminando el elemento canvas donde se grafica el streaming de video
el uso de CPU se redujo a un 0.5\% en Firefox y 30\% en Google Chrome,
por lo que aparentemente ese uso de CPU se encuentra relacionado
con la decodificación y renderizado de video.

\section{Pruebas con la interfaz Javascript}

Estas pruebas llevaron a algunas modificaciones de la interfaz
web, la API Javascript y el servidor en base a la experiencia de uso,
por ejemplo:
\begin{itemize}
    \item Se incorporó la funcionalidad al servidor que cuando
        el cliente se desconecta del WebSocket, se liberan todas
        sus reservas. Esto es una cuestión de gustos, podría
        mantenerse también la reserva sin problema hasta su
        vencimiento, pero haciendo pruebas con una cantidad
        limitada de robots se hizo notorio que un sistema
        así desperdicia mucho el tiempo ocioso del robot.
    \item Dado que al cerrar el WebSocket se pierde la reserva
        se hizo que también se envíe el mensaje \texttt{stop}
        a cada robot reservado por el usuario, de manera
        que el robot no se quede en movimiento una vez
        terminada la reserva.
    \item Se incorporó una caja con tips debajo de la vista de
        la cámara, en la misma van rotando una serie de consejos
        acerca de aspectos de la API Javascript que pueden
        resultar poco familiares a una persona que no utilize este
        lenguaje con frecuencia y otros muy específicos
        del control de los robots con XRemoteBot.
    \item Se modificó la forma de evaluar el código, para que se
        ejecute en un contexto donde las funciones
        \texttt{setInterval()} y \texttt{setTimeout} se encuentran
        ocultas por otras implementaciones que garantizan que cada
        vez que se presione el botón ``Ejecutar'' se eliminen
        todos los timeouts e intervalos previamente creados.
        De otra manera si el usuario escribe un script que usa
        estas funciones
        (cuestión que parece ser conveniente en este entorno)
        y luego escribe otro, los timeouts e intervalos
        de la versión anterior se seguirían ejecutando
        interfiriendo con el comportamiento del script actual.
\end{itemize}

\subsection{Pruebas de codificación}

Dado el uso de Promises en el diseño de este cliente, fue
necesario repensar la forma
de resolver los ejercicios habituales. Por ejemplo en el
código~\ref{lst:movimientos_js} se ve un script que mueve el
robot,
código~\ref{lst:esquivar_js} se ve un algoritmo para que el robot
avance esquivando obstáculos y en el código~\ref{lst:random_js}
muestra un script que mueve el robot en direcciones aleatorias.

\begin{lstlisting}[language=C,
caption={Secuencia de movimientos},
label=lst:movimientos_js]
var server = new Server('ws://xremotebot.example:8000/api',
                        'api_key');
server.onConnect(function(){
	server.fetch_robot().then(function(robot_obj){
		var robot = new Robot(server, robot_obj);
        robot.forward(50, 5);
        robot.turnLeft(50, 2);
        robot.turnRight(50, 2);
        alert('Hola');
	});
});
\end{lstlisting}

\begin{lstlisting}[language=C,
caption={Esquivar obstáculos},
label=lst:esquivar_js]
var server = new Server('ws://xremotebot.example:8000/api',
                        'api_key');
server.onConnect(function(){
    server.fetch_robot().then(function(robot_obj){
        var robot = new Robot(server, robot_obj);
        var esquivar = function(){
            robot.getObstacle().then(function(obstacle){
                if (obstacle){
                    robot.backward(40, 1);
                    if (Math.random() < 0.5){
                        robot.turnLeft(40, 1);
                    }
                    else{
                        robot.turnRight(40, 1);
                    }
                    robot.forward(40);
                }
                setTimeout(esquivar, 500);
            });
        };
        robot.forward(40);
        setTimeout(esquivar, 500);
    });
});
\end{lstlisting}

Al ejecutar el código~\ref{lst:movimientos_js} se puede ver que la
función \texttt{alert()} se está ejecutando mucho antes que el
robot termine de moverse, de hecho lo hace de manera casi instantánea
al presionar ``Ejecutar''. Esto es porque
si bien se implementó un mecanismo donde el cliente
antes de enviar cada mensaje espera a que el servidor conteste el
anterior para simular una ejecución secuencial y para simular
funciones bloqueantes que demoran la ejecución del script, realmente
el cliente no deja de ser asincrónico.
La ejecución de \texttt{forward()} y cada uno de los
métodos del robot en realidad solamente instancia y retorna una
``promesa'', por lo que el tiempo que tardan en ejecutarse
estos movimientos es insignificante, pero vemos que el robot
se mueve como si cada uno de estos métodos congelara la ejecución
del script por este mecanismo que tiene soporte tanto del lado
del cliente como del lado del servidor que demora la respuesta
de algunos métodos apropósito.

\begin{lstlisting}[language=C,
caption={Secuencia de movimientos},
label=lst:movimientos_then_js]
var server = new Server('ws://xremotebot.example:8000/api',
                        'api_key');
server.onConnect(function(){
	server.fetch_robot().then(function(robot_obj){
		var robot = new Robot(server, robot_obj);
        robot.forward(50, 5).then(function(){
            robot.turnLeft(50, 2);
        }).then(function(){
            robot.turnRight(50, 2);
        }).then(function(){
            alert('Hola');
        });
	});
});
\end{lstlisting}

Es posible hacer una versión donde \texttt{alert()} se ejecute cuando
el robot termine de moverse, pero el código de esta versión
(código~\ref{lst:movimientos_then_js}) es
un poco más engorroso ya que tiene muchos niveles de anidamiento.

En el código~\ref{lst:esquivar_js} a simple vista se ven diferencias
significativas de la forma de codificar con esta API en comparación
con DuinoBot e incluso en comparación con los otros clientes de
XRemoteBot. Lo más importante es que dado que los métodos retornan
``promesas'' no se puede usar un \texttt{while} o \texttt{for} para
repetir una acción por ejemplo mientras no hay obstáculos, por lo que
hay que recurrir a \texttt{setInterval()} o bien a \texttt{setTimeout()}
para tener más control.

El código~\ref{lst:random_js} usa la misma estrategia que el ejemplo
anterior usando \texttt{setInterval()} para realizar una tarea
que con DuinoBot en Python se haría con un \texttt{while} y además
accede a la API Math del navegador.

\section{Pruebas con los clientes Python y Ruby}
Estos clientes tienen una complejidad menor que el cliente Javascript
por lo que fueron rápidamente probados sin mayores dificultades.

La API de estos clientes se asemeja tanto a la API original de
DuinoBot que no fue necesario repensar como hacer las actividades
típicas con los robots como lo fue en la versión Javascript.
