\chapter{Pruebas}\label{cha:pruebas}

El servidor se probó con Python 2.7 (portarlo a Python 3 involucraría
portar DuinoBot, Myro y otros módulos asociados) en entornos creados
con virtualenv\footnote{FIXME REF} y con una base de datos
Sqlite\footnote{FIXME REF}
sobre
sistemas Lihuen 5 y 6 beta (basados en Debian Wheezy y Jessie respectivamente).
Las dependencias se instalaron declarandolas en \texttt{setup.py} e instalandolas
con pip, a excepción de los módulos DuinoBot y Myro que fueron instalados
desde repositorios Git usando también pip, pero indicando la URL manualmente
ya que no fué posible declararlos de esta manera en \texttt{setup.py}.

\begin{lstlisting}[language=bash,caption={Instalación de XRemoteBot},label=]
git clone https://github.com/fernandolopez/xremotebot.git
cd xremotebot
virtualenv .
. bin/activate
pip install -r requirements.txt
pip install 'git+https://github.com/Robots-Linti/duinobot.git@pygame_opcional'
pip install 'git+https://github.com/fernandolopez/Myro.git'
\end{lstlisting}

Para soportar el streaming de video además es necesario 



\begin{lstlisting}[language=bash,caption={Instalación de XRemoteBot},label=]
./run.sh
\end{lstlisting}

usando los archivos requirementes.txt
y requirements-dev.txt, a excepción de los módulos DuinoBot y Myro que se
instalaron con pip desde Git por lo que no fue posible usar los archivos



