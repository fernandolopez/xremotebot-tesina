%% This is an example first chapter.  You should put chapter/appendix that you
%% write into a separate file, and add a line \include{yourfilename} to
%% main.tex, where `yourfilename.tex' is the name of the chapter/appendix file.
%% You can process specific files by typing their names in at the 
%% \files=
%% prompt when you run the file main.tex through LaTeX.
\chapter{Introduction}

XRemotebot es un servidor Web que provee una API JSON para interactuar
con robots didácticos de forma remota y compartiendo un único enlace
físico con los robots. Está implementado en Python usando principalmente el
framework web Tornado, SQLAlchemy para acceder a la base de datos, Celery
para manejar las tareas que pueden generar una demora innecesaria en el
procesamiento de requerimientos HTTP y los módulos DuinoBot y Myro para
la comunicación con los robots. El cliente Javascript se encuentra
integrado con el servidor y la intensión es que sea utilizado desde una
vista web provista por el mismo, mientras que los clientes Python y Ruby
son independientes y la intensión es que sean utilizados en programas
tradicionales que se ejecuten desde fuera del navegador para enseñar
a programar en estos lenguajes usando un entorno habitual. XRemotebot
es un rediseño y una reescritura completa de Remotebot 1, un servidor
simple realizado como aplicación auxiliar de una aplicación Android
desarrollada
como trabajo práctico para la materia Laboratorio de Software.

El capítulo 1 relata la motivación para el desarrollo de XRemotebot
y alternativas similares en existencia en la actualidad.

El capítulo 2 describe la arquitectura de DuinoBot, Myro, Remotebot y
XRemotebot detallando las mejoras provistas por XRemotebot.

El capítulo 3 describe el protocolo de capa de aplicación diseñado para
este servidor y las modalidades de operación del servidor.

El capítulo 4 describe la implementación de los clientes y el modo de uso
de cada uno destacando algunas diferencias y desiciones de diseño en la
implementación en Javascript.

El anexo A muestra las pruebas realizadas para seleccionar un método
de serialización para el protocolo de capa de aplicación desarrollado.


\section{Motivación del diseño y desarrollo de XRemotebot}\label{ch1:motivacion}

Desde hace años desde la Secretaría de Extensión de la Facultad de Informática
se brindan cursos de programación en Python para alumnos de escuelas
secundarias utilizando robots didácticos, aunque se han utilizado a la fecha
dos modelos distintos de robots en escencia las características principales
de ambos son las mismas y coinciden con las de otros robots usados en la
Argentina para enseñar a programar, a saber:
\begin{itemize}
    \item El medio de locomoción es con 2 motores continuos que mueven cada
        uno una de las ruedas laterales.
    \item Cuentan con algún sensor o sensores que permiten detectar obstáculos.
    \item Cuentan con algún sensor o snesores que permiten detectar líneas.
    \item Operan sin el uso de cables
\end{itemize}

Cabe destacar sobre este último ítem que algunos de los robots no requieren
cables para operar ya que son programados con anterioridad a través de un
cable (usualmente USB o Serial) normalmente en algún lenguaje como C, Assembler
o Basic. Pero los robots usados durante los cursos
de Python no requieren cables ya que son controlados a través de señales
inalámbricas lo que permite controlarlos en tiempo real y utilizando
un intérprete Python estándar (cPython) instalado en el dispositivo
controlante.

Dado el costo y fragilidad de los robots habitualmente los alumnos interactúan
con los mismos solamente en el aula, ya sea en su escuela si la misma tuvo
el privilegio de adquirir los robots o en nuestra Facultad si el alumno
realiza alguna pasantía en la misma. Esto tiene varias consecuencias:
\begin{itemize}
    \item El alumno al estar en una situación formal en la escuela
        compartiendo el recurso límitado que es el robot con otros
        alumnos posiblemente no tenga el tiempo o el ambiente más apropiado
        para experimentar de forma lúdica con el robot.
    \item La tarea para casa solamente es realizable a través de un simulador
        que puede ser lo suficientemente completo y fiel como para aprender
        a programar, pero resulta menos estimulante y realista que manipular
        un robot real.
    \item Alumnos de escuelas que no tienen los recursos necesarios para
        adquirir el equipamento o de escuelas alejadas de la plata que
        no tienen la posibilidad de realizar una pasantía en nuestra
        Facultad no pueden interactuar con robots reales
        (a lo sumo podrán usar un simulador).
\end{itemize}

Por otro lado los dongles XBee que permiten conectarse con los robots
Multiplo N6
son relativamente costosos, el esquema normal de conexionado entre dispositivos
controladores y robots descripto en el capítulo~\ref{cha:arquitectura} requiere
2 dogles XBee por robot, uno conectado directamente al robot y el otro
conectado por USB al dispositivo controlante. En consecuencia:
\begin{itemize}
    \item El dispositivo controlante debe tener un puerto USB y los drivers
        necesarios para detectar la interfaz serial con el XBee, esto deja
        fuera de juego celulares y tablets.
    \item El costo de operar cada robot en este esquema es sensiblemente
        superior al costo que tendría si varios alumnos pudieran
        controlar varios robots usando un solo dispositivo XBee compartido
        entre varios dispositivos controlantes.
\end{itemize}

Tomando estos dos grupos de problemáticas intenté implementar una solución
que permita
superarlos de forma simultánea, esta solución permite controlar los
robots a través de una red como puede ser Internet permitiendo a los alumnos
conectarse a los robots desde sus hogares y a las instituciones a compartir
el uso de sus robots. Por otro lado es posible también usar el servidor
en un ámbito local deshabilitando la necesidad de requerir un login
para poder compartir un solo dispositivo XBee entre varios alumnos.

Como consecuencia de estos requerimientos quedaba claro que el servidor
debía tener una interfaz web y proveer acceso concurrente con relativa
baja latencia para permitir a múltiples alumnos acceder a múltiples robots
al mismo tiempo. Tener un servidor web basado en protocolos estándar hizo
que el requerimiento de que el cliente estuviera escrito en Python fuera
artificial, lo que me hizo pensar en aprovechar la oportunidad para
implementar otros clientes que permitieran usar los robots en cursos
de programación de distintos lenguajes sin reimplementar el protocolo
de bajo nivel, la falta de necesidad de acceder al hardware directamente
por una interfaz serial a través del USB habilita a la posiblidad de
implementar un cliente Javascript que se ejecute en un navegador web
normal y el hecho de no requerir reimplementar el protocolo de bajo nivel
habilita a que este servidor pueda ser fácilmente adaptado para controlar
otros robots de naturaleza similar a los utilizados en los cursos ya
descriptos.

El servidor Remotebot original tenía algunas de estas características, pero
estaba probremente implementado ya que no era más que una herramienta auxiliar
para un cliente muy específico y era meramente un intermediario entre el
cliente original implementado en Java y el módulo Duinobot. Este servidor
estaba severamente limitado
ya que no era configurable, no contaba con ninguna forma de visualizar a los
robots de forma remota, no permitía autenticación, no disponía un sistema
de reserva de robots por lo que un cliente podía interferir en la operación
de un robot de otro cliente y las operaciones bloqueantes de un cliente
impedían el uso de los robots al resto de los clientes hasta que esa operación
terminase.

\section{Alternativas similares a XRemotebot}

\subsection{Educabot}

El proyecto Educabot tiene por objetivos enseñar tecnología a niños a través
del uso, programación y construcción de robots.

En la parte de construcción este proyecto plantea un modelo de robot denominado
``Rolo'' para los jóvenes de más de 10 años, mientras que para los más chicos
se plantean actividades con el robot ``elBrian'' que consisten en programarlo
a través de un lenguaje de programación visual usando bloques y también
controlarlo a través de una interfaz web que muestra las imágenes emitidas
por la cámara incorporada en este robot y además permite controlarlo con
botones en pantalla que determinan en qué dirección debe moverse el robot.
Las tecnologías utilizadas en la interfaz web de ``elBrian'' coinciden en gran
parte con las utilizadas en el desarrollo de XRemotebot, pero el objetivo del
servidor es controlar un único robot en un ambiente local y además este servidor
se instala en el robot cuestión que sería imposible en los robots basados en
microcontroladores AVR y Parallax a los que XRemotebot se encuentra dirigido.

El servidor web de ``elBrian'' está implementado usando el framework Tornado,
Websockets, mjpg-streamer, opencv y JSON. El mismo está diseñado para ejecutarse
en el robot ya que el mismo está basado en una placa RaspberryPi, la cuál
cuenta con un procesador ARM perfectamente capaz de ejecutar un sistema
operativo completo como GNU/Linux y de soportar el intérprete oficial de Python.

Mientras que este servidor coincide en gran medida en la elección de lenguaje
y bibliotecas utilizadas su implementación es específica para el robot ``elBrian''
y no podría ser portada para robots con menores capacidades de procesamiento
sin una reescritura significativa. Además el protocolo utilizado no contempla
el acceso a valores de sensores, los únicos mensajes que permite enviar
al robot son movimientos.

\begin{itemize}
    \item Educabot \url{http://www.educabot.org/}
    \item Código fuente de ``elBrian'' \url{https://github.com/educabot/elBraian}
\end{itemize}

\subsection{Gobot con cppp-io}

Gobot es una biblioteca que permite controlar robots programando en el lenguaje
Go, esta biblioteca soporta el protocolo Firmata para controlar robots
conectados directamente a través de una interfaz serial, como es el caso
de los robots Multiplo N6, y soporta la API cppp-io que define una API JSON
que permite el acceso a la información y control de robots a través de la Web.

Gobot además tiene compatibilidad con distintos sensores y robots, además de
placas utilizadas normalmente en la construcción de robots como Arduino,
Raspberry Pi, Intel Edison y Beaglebone Black.

Este proyecto es interesante como base para desarrollar algún proyecto
similar a XRemotebot en Go, pero requeriría además la reimplementación
del módulo de Python DuinoBot que controla, a través de una versión
modificada del protocolo Firmata, a los robots Multiplo N6 y por otro
lado los robots Scribbler tampoco aparecen en la lista de robots soportados.

\begin{itemize}
    \item Gobot \url{http://gobot.io}
    \item Especificación de cppp-io \url{https://github.com/hybridgroup/cppp-io/}
\end{itemize}

\subsection{Tele Toyland}

Este sitio provee acceso a varios dispositivos a través de una interfaz web,
por ejemplo es posible controlar un cabezal con una punta que dibuja sobre
una caja de arena, basta con hacer clic sobre las posiciones sobre las cuales
se quiere que pase la punta y presionar el botón ``go'' para que el cabezal
empiece a moverse dibujando lo pedido, en este y el resto de los experimentos
disponibles en el sitio los resultados se pueden ver a través de un streaming
de video.

El sitio no provee detalles del software, ni el protocolo utilizado.

\begin{itemize}
    \item \url{http://www.teletoyland.com}
\end{itemize}

% FIXME
%\subsection{}
%http://www3.uji.es/~pnebot/Files/Articuls/RemoteProgramming.pdf
%Otro
%http://telerobot.mech.uwa.edu.au/Telerobot/instructions.html
\subsection{DIY}
% www.linuxuser.co.uk/tutorials/control-your-raspberry-pi-robot-from-a-web-connected-device
Finalmente en la consigna de ``do it yourself'' existen diversas guias para
programar servidores que permitan controlar robots o microcontroladores
en general, se puede encontrar un caso muy bien explicado en el sitio
de Adafruit~\url{https://learn.adafruit.com/wifi-controlled-mobile-robot/building-the-web-interface},
este es un buen ejercicio de programación, sobre todo para aprender a
programar servidores que provean una API web y clientes que la consuman. Sin
embargo estas guías son introductorias y el objetivo es crear un servidor
muy simple, similar a lo que fue el servidor Remotebot, pensados para ser
usados en un ambiente local ya que no proveen autenticación en general.

