% Apéndice A Serialización

\chapter{Protocolo}
\label{ch:protocolo}

Para permitir el uso de los robots sin la necesidad de instalar librerías
especiales en los clientes se eligió implementar el sistema como una
aplicación web. Sin embargo, como se menciona en el
capítulo~\ref{ch:serializacion} no se utiliza HTTP para el intercambio de
mensajes y valores de retorno entre los clientes y el servidor sino
WebSocket.

El protocolo WebSocket permite mantener conexiones persistentes y no envía
encabezados HTTP en cada mensaje, reduciendo así el overhead en los mensajes
intercambiados que supondría el uso de HTTP~\cite{Wang}.

La elección de este protocolo trae como desventaja frente al uso de HTTP la
necesidad de tener consideraciones de seguridad extra tales como verificar
el encabezado \texttt{Origin} en el
% FIXME: traducir hanshake...
\textit{initial HTTP WebSocket handshake}~\cite{Grigorik} e implementar
mecanismos de
autenticación en la aplicación~\cite{Wang} para evitar
ataques del tipo CSRF (Cross-Site Request
Forgery)~\cite{OWASP-2014}~\cite{Schneider-2013}.

% http://www.christian-schneider.net/CrossSiteWebSocketHijacking.html
% https://www.owasp.org/images/5/52/OWASP_Testing_Guide_v4.pdf
