% Apéndice A Serialización

\chapter{Serialización}
\label{ch:serializacion}

Siendo una aplicación cliente-servidor remotebot requiere algún método de
serialización para intercambiar datos entre los clientes y el servidor,
considerando que el servidor no es más que un servidor web que usa websockets
como protocolo, el método de serialización más adecuado parece ser JSON
(Javascript Serialization Object Notation).

JSON cuenta con las siguientes características:

\begin{enumerate}
    \item Es un formato estandarizado por ECMA~\citep{ecma-404}
        y está especificado por la rfc-7159~\citep{rfc-7159}.
    \item Soporta los tipos de datos necesarios para intercambiar mensajes con
        los datos necesarios para controlar los robots usando un nivel de
        abstracción adecuado.
    \item Al ser un formato de texto es simple analizar el tráfico entre los
        clientes y el servidor para detectar posibles errores.
    \item Está soportado de forma nativa por los navegadores más
        utilizados%~\footnote{\url{http://caniuse.com/#search=json}}.
\end{enumerate}

Pero también existen otras alternativas cuyos objetivos son serialización
y de-serialización rápida, y datos serializados más compactos
como BSON (Binary JSON) y (CBOR Concise Binary Object Representation).

Algunas características de BSON y CBOR son:

\begin{enumerate}
    \item Codifican la información en formato binario.
    \item Son tan fáciles de usar como JSON.
    \item Ambos formatos proveen un superset de los tipos de datos provistos
        por JSON.
    \item CBOR está especificado por la rfc-7049~\citep{rfc-7049}.
    \item BSON tiene una especificación informal pero cuenta una
        implementación bien conocida siendo la representación de datos primaria en
        MongoDB~\footnote{\url{http://bsonspec.org/}}.
    \item Por estar en formato binario decodificar una captura de tráfico
        para depurar un programa es más laborioso.
    \item En algunos casos estos formatos binarios generarán mensajes más
        chicos que JSON, pero no siempre.
    \item Requieren usar librerías Javascript en los clientes web ya que los
        navegadores no lo implementan de forma nativa.
\end{enumerate}

Para tomar un decisión respecto al formato a utilizar se decidió generar
16 archivos \texttt{JSON} con datos aleatorios, estos archivos se cargaron
con una cantidad de entradas múltiplo de 1024 desde 1024 hasta 16384, donde
cada una de estas entradas es un objeto \texttt{JSON} con 5 entradas de tipo
numérico, 5 strings, 5 objetos (cada uno con una entrada numérica) y 5
arrays (cada uno con 2 entradas de tipo string).

Estos archivos se cargaron desde scripts similares en Python y Javascript que
de-serializaron y serializaron estos datos repetidas veces calculando el tiempo
promedio que llevó hacer cada una de estas acciones para cada formato.

Se eligió hacer las pruebas con Python y Javascript porque el primero es el
lenguaje de implementación del servidor y del cliente que probablemente tenga
más uso en un futuro, mientras que Javascript es el lenguaje de implementación
del cliente que se ejecutará en los navegadores web y que puede servir como
base para implementaciones con Brython, Skulp, Opal, etc...

Para hacer el experimento repetible la implementación en
Javascript se ejecutó con el intérprete \textbf{nodejs} basado en
el motor \textbf{v8} usado por \textbf{Chrome} y además se crearon
scripts de apoyo para generar los datos de prueba, ejecutar los scripts
con los distintos formatos de forma automatizada y formatear los datos
resultantes en archivos csv.

Todas las pruebas se realizaron sobre Lihuen 6 beta (basado en Debian Jessie)
en una notebook con procesador ``Intel(R) Core(TM) i3 CPU M 370 @ 2.40GHz''
y 4GB de RAM % NOTA: son 4GB no 4GiB
usando Python 3.4.2 y NodeJS 0.10.35~\footnote{Se usó NodeJS 0.10.35 ya que
la versión 0.10.29 distribuida al momento con Debian Jessie tenía un bug que
generaba un error de memoria al
deserializar strings largos con \texttt{JSON.load} (CVE-2014-5256)}

De estas pruebas se desprenden las mediciones de las
figuras~\ref{fig:ser-time-py}, \ref{fig:ser-time-js} y \ref{fig:ser-size}.

\begin{figure}
    \centering
    \begin{framed}
        \begin{tikzpicture}[trim axis left, trim axis right]
            \begin{axis}[
                    height=10cm,
                    width=10cm,
                    scaled x ticks = false, % No usar notación exponencial
                    xlabel=Cantidad de entradas,
                    ylabel=Segundos,
                    x tick label style={/pgf/number format/fixed, rotate=45, anchor=north east}, % Las etiquetas son números y rotar 45°
                    xtick=data, % Generar una etiqueta por punto en los datos
                    ylabel near ticks, % Poner las etiquetas más o menos cerca de algunos datos
                    ymin=0, % No hay tiempos negativos
                    xmin=1024, % La escala empieza en 1024
                    legend style={legend pos=north west}, % Cuadro de leyendas a la derecha arriba
                    every axis x label/.style={
                        at={(axis description cs:0.5,-0.15)}, % Desplazar la descripción del eje x
                        anchor=north, % El desplazamiento relativo a la parte superior del texto
                    },
                    legend cell align=left, % Alineación del texto en la caja de leyendas
                ]
                \foreach \s in {json, bson, cbor} {
                    \addlegendentryexpanded{\s\ dump}
                    \addplot table [x=entries, y=dump_time, col sep=comma] {data/py-\s.csv};

                    \addlegendentryexpanded{\s\ load}
                    \addplot table [x=entries, y=load_time, col sep=comma] {data/py-\s.csv};
                }
            \end{axis}
        \end{tikzpicture}
    \end{framed}

    \caption{Cantidad de entradas versus tiempo de serialización y deserialización en Python}
    \label{fig:ser-time-py}
\end{figure}

\begin{figure}
    \centering
    \begin{framed}
        \begin{tikzpicture}[trim axis left, trim axis right]
            \begin{axis}[
                    height=10cm,
                    width=10cm,
                    scaled x ticks = false, % No usar notación exponencial
                    xlabel=Cantidad de entradas,
                    ylabel=Segundos,
                    x tick label style={/pgf/number format/fixed, rotate=45, anchor=north east}, % Las etiquetas son números y rotar 45°
                    xtick=data, % Generar una etiqueta por punto en los datos
                    ylabel near ticks, % Poner las etiquetas más o menos cerca de algunos datos
                    ymin=0, % No hay tiempos negativos
                    xmin=1024, % La escala empieza en 1024
                    legend style={legend pos=north west}, % Cuadro de leyendas a la derecha arriba
                    every axis x label/.style={
                        at={(axis description cs:0.5,-0.15)}, % Desplazar la descripción del eje x
                        anchor=north, % El desplazamiento relativo a la parte superior del texto
                    },
                    legend cell align=left, % Alineación del texto en la caja de leyendas
                ]
                \foreach \s in {json, bson, cbor} {
                    \addlegendentryexpanded{\s\ dump}
                    \addplot table [x=entries, y=dump_time, col sep=comma] {data/js-\s.csv};

                    \addlegendentryexpanded{\s\ load}
                    \addplot table [x=entries, y=load_time, col sep=comma] {data/js-\s.csv};
                }
            \end{axis}
        \end{tikzpicture}
    \end{framed}
    \caption{Cantidad de entradas versus tiempo de serialización y deserialización en Javascript (nodejs)}
    \label{fig:ser-time-js}
\end{figure}


\begin{figure}
    \centering
    \begin{framed}
        \begin{tikzpicture}[trim axis left, trim axis right]
            \begin{axis}[
                    height=10cm,
                    width=10cm,
                    scaled x ticks = false,
                    xlabel=Cantidad de entradas,
                    ylabel=Tamaño en MiB,
                    x tick label style={/pgf/number format/fixed, rotate=45, anchor=north east},
                    xtick=data,
                    ylabel near ticks,
                    ymin=0,
                    xmin=1024,
                    legend style={legend pos=north west},
                    every axis x label/.style={
                        at={(axis description cs:0.5,-0.15)},
                        anchor=north,
                    },
                    legend cell align=left
                ]
                \foreach \s in {json, bson, cbor} {
                    \addlegendentryexpanded{\s}
                    \addplot table [x=entries, y=serialized_size, col sep=comma] {data/js-\s.csv};

                }
            \end{axis}
        \end{tikzpicture}
    \end{framed}
    \caption{Cantidad de entradas versus tamaño del archivo serializado}
    \label{fig:ser-size}
\end{figure}

De la figura~\ref{fig:ser-size} se desprende que los distintos métodos de
serialización no ofrecen diferencias significativas en el tamaño de los
strings generados para los volumenes de datos probados. Por otro lado
se puede observar en la figura~\ref{fig-ser-time-py} que al menos en
Python y con la librería usada CBOR tiene los mejores tiempos, sin
embargo en las pruebas con Javascript en la figura~\ref{fig:ser-time-js}
se puede observar que los mejores tiempos son para el formato JSON,
esto es entendible ya que este es el único formato procesado de forma
nativa~\footnote{Si bien es posible encontrar parsers nativos para los
otros formatos y usarlos desde NodeJS no se utilizó esta modalidad porque
no sería una prueba realista dado que los motores Javascript de los
navegadores no permiten esto.}.

Teniendo en cuenta que las diferencias de tiempo al procesar los datos
en Python son, en comparación con la versión en Javascript, marginales
(apenas alrededor de $0.25$ segundos para $2^{14}$
entradas entre el mejor y peor tiempo de deserialización) y las diferencias
en cuanto a tamaño del string generado al
serializar también son pequeñas (aproximadamente 2MiB de diferencia para
$2^{14}$ entradas entre el mejor y el peor método
