\chapter{WebSockets y Promises}\label{cha:websockets_y_promises}

Para el desarrollo de este trabajo decidió utilizar dos tecnologías
que aún están en proceso de estandarización pero ya son ampliamente
soportadas por distintos navegadores
Web%
~\footnote{\url{http://caniuse.com/\#feat=websockets}}%
~\footnote{\url{http://caniuse.com/\#feat=promises}}
como son la API
\textit{WebSocket}~\footnote{La API en proceso de estandarización por la W3C,
pero el protocolo ya fue estandarizado por IETF.}
y la API \textit{Promise}~\footnote{En proceso de estandarización para
ECMAScript 6 (Harmony).}.

En este capítulo se presenta una breve reseña de ambos y los motivos
por los cuales fueron elegidos.

\section{WebSockets}

Para permitir el uso remoto de los robots a través de Internet sin necesidad
de requerir configuraciones ni puertos especiales se eligió implementar el
sistema como una
aplicación web. Sin embargo, como se menciona en el
capítulo~\ref{cha:serializacion} no se utiliza HTTP para el intercambio de
mensajes y valores de retorno entre los clientes y el servidor sino el
protocolo WebSocket.

El protocolo WebSocket permite mantener conexiones persistentes y no envía
encabezados HTTP en cada mensaje, reduciendo así el overhead en los mensajes
intercambiados que supondría el uso de HTTP~\citep{wang_2013}, aprovechando al
mismo tiempo los puertos TCP que abre el servidor Web.

La elección de este protocolo trae como desventaja frente al uso de HTTP la
necesidad de tener consideraciones de seguridad especiales, para la parte
Web del sistema se utiliza autenticación con cookies pero no resulta seguro
utilizar el mismo mecanismo para la sesión con WebSockets ya que el sistema
quedaría vulnerable a ataques de tipo CSRF (Cross-Site Request
Forgery)~\citep{owasp_2014}~\footnote{
\url{https://www.owasp.org/images/5/52/OWASP_Testing_Guide_v4.pdf}}.

Una de las estrategias recomendadas es verificar el encabezado \texttt{Origin}
desde el servidor, pero esta técnica no es suficiente ya que los
clientes pueden falsear este campo y solamente si el cliente es un navegador
envía este encabezado.
Otra estrategia es tener un mecanismo de autenticación
separado para la comunicación~\footnote{\url{https://www.christian-schneider.net/CrossSiteWebSocketHijacking.html}},
esta fue la estrategia elegida para XRemoteBot usando una \textit{API key}
para identificar a los usuarios al inicio de la comunicación.

\section{Promises}

La API de WebSockets disponible en los navegadores Web requieren
redefinir una serie de funciones de la instancia del WebSocket
para recibir y enviar mensajes~\citep{websocket_2014}, a saber:
\begin{description}
    \item[\texttt{WebSocket\#onopen()}] se ejecutará cuando
    la conexión esté establecida.
    \item[\texttt{WebSocket\#onerror()}] se ejecutará ante un error en
    la conexión.
    \item[\texttt{WebSocket\#onclose()}] se ejecutará si la conexión
    se cierra.
    \item[\texttt{WebSocket\#onmessage()}] se ejecutará al recibir un
    mensaje desde el servidor, recibe como argumento el mensaje
    enviado por el servidor.
\end{description}

La intención del autor es ocultar estos detalles de implementación de
los usuarios que quieran controlar los robots usando la API Javascript,
para esto es posible utilizar el objeto \textit{Promise} que se encuentra
en proceso de estandarización para ECMAScript 6, pero ya se puede
usar en los navegadores más populares.

Los objetos \textit{Promise} se utilizan para obtener valores resultantes
de cómputos asincrónicos, como es el caso de la respuesta a una petición
hecha con WebSockets, la interfaz de \textit{Promise} no permite programar
los robots en Javascript con una interfaz idéntica a la usada al programar
los robots con la biblioteca DuinoBot, pero es relativamente fácil de
utilizar y ante la imposibilidad de demorar la ejecución del código
Javascript sin crear funciones adicionales provee una alternativa
relativamente simple en comparación con el uso directo de la interfaz
de los objetos \texttt{WebSocket}.

Los objetos \textit{Promise} se instancian pasándoles como argumento una
función que a su vez recibe 2 argumentos, \texttt{resolve} y \texttt{reject}.
El primer argumento ``cumple'' o ``resuelve'' la promesa y el segundo la
``rechaza''.
Las ``promesas'' tienen dos métodos: \texttt{then} y \texttt{catch} que se
ejecutan cuando la promesa se ``cumple'' o se ``rechaza'' respectivamente.

XRemoteBot para Javascript al crear cada mensaje le asigna un \texttt{msg\_id},
crea una \texttt{Promise} y guarda en un \texttt{object} de mensajes
pendientes un objeto que contiene las funciones \texttt{resolve} y
\texttt{reject} asociados con esa \texttt{Promise}
usando como clave el \texttt{msg\_id}. Al recibir una respuesta
desde el servidor, el cuerpo de \texttt{WebSocket\#onmessage()} toma el
\texttt{msg\_id} de la respuesta, lo utiliza para recuperar las funciones
asociadas con la \texttt{Promise} de este mensaje y ejecuta el
\texttt{resolve} pasándole como argumento la respuesta del servidor.
De esta manera las ``promesas'' asociadas con cada
mensaje se ``cumplen'' al obtener la respuesta correspondiente del servidor.

Cada método de XRemoteBot para Javascript que involucre enviar un mensaje
retornará una ``promesa'' que se cumplirá cuando el servidor responda
el mensaje.

Esto permite, por ejemplo, obtener el valor del sensor de distancia de un 
robot de forma simple como se ve en el código~\ref{lst:promises_get_obstacle}.
Si bien esta interfaz no es idéntica a la de DuinoBot, es lo más parecido
que se pudo lograr.

\begin{lstlisting}[language=C,
caption={Implementación de \texttt{getObstacle()} con ``promesas''},
label=lst:promises_get_obstacle]
robot.getObstacle().then(function(hay_obstaculo){
    if (hay_obstaculo.value){
        console.log("Hay un obstáculo al frente");
    }
    else {
        console.log("No hay obstáculos");
    }
});
\end{lstlisting}
